\documentclass[a4paper,12pt]{article}
\usepackage[utf8]{inputenc}
\usepackage{amsmath}
\usepackage{amsfonts}
\usepackage{amssymb}
\usepackage{graphicx}
\usepackage{braket}

\numberwithin{equation}{section}
\renewcommand\thesubsection{\alph{subsection}}
\newcommand{\bvp}[1]{\mathbf{#1}'}
\newcommand{\bv}[1]{\mathbf{#1}}

%opening
\title{Statistical Mechanics II HW2}
\author{Vincent Baker}

\begin{document}

\maketitle

\section{Problem 1}
We want to show the $p_i=\frac{1}{N}$ maximizes the total entropy. 
With $S(\{p_i\})=p_i\ln{p_i}$, we maximize $\sum_{i=1}^Np_i\ln{p_i}$ subject to $\sum p_i = 1$.
Using the method of Lagrange multipliers:
\begin{gather}
 \sum_{i=1}^Np_i\ln{p_i}+\lambda \left(\sum_{i=1}^Np_i \right)=0\\
 \frac{d}{dp_i}=p_i+1+\lambda=0\\
 p_i=-1-\lambda\\
 \sum p_i - 1=0
\end{gather}
From 1.3 and 1.4 we find:
\begin{gather}
 -N-N\lambda-1=0\\
 \lambda=-\frac{1}{N}-1
\end{gather}
Substituting back into 1.2:
\begin{gather}
 p_i+1-\frac{1}{N}-1=0\\
 p_i=\frac{1}{N}
\end{gather}

\section{Problem 2}
Solve $Ax^2+By^2$ subject to $ax+by-c=0$.
\begin{gather}
 g(x,y)=Ax^2+By^2+\lambda(ax+by-c)=0\\
 \frac{\partial g}{\partial x}=2Ax+\lambda a=0,\ x=-\frac{\lambda}{2}\frac{a}{A}\\
 \frac{\partial g}{\partial y}=2By+\lambda b = 0,\ y=-\frac{\lambda}{2}\frac{b}{B}\\
 -\frac{\lambda}{2}\left(\frac{a^2}{A}+\frac{b^2}{B}  \right)-c=0\\
 x=\frac{c  \frac{a}{A}}{\frac{a^2}{A}+\frac{b^2}{B}}\\
 y=\frac{c  \frac{b}{B}}{\frac{a^2}{A}+\frac{b^2}{B}}
\end{gather}

\section{Problem 3}
We find the probabilities that maximize total entropy constrained by an expectation relation.
The expectation relation $<N>=\frac{2}{7}$ creates the constraint $p_1+2p_2=\frac{2}{7}$.
\begin{gather}
 g(p_i)=\sum_i p_i\ln(p_i) + \lambda_1(p_0+p_1+p_2-1)+\lambda_2(p_1+2p_2-\frac{2}{7})=0\\
 \frac{\partial g}{\partial p_0} = \ln{p_0}+1+\lambda_1=0,\ p_0=e^{-(1+\lambda_1)}\\
 \frac{\partial g}{\partial p_1} = \ln{p_1}+1+\lambda_1+\lambda_2=0,\ p_1=e^{-(1+\lambda_1+\lambda_2)}\\
 \frac{\partial g}{\partial p_2} = \ln{p_2}+1+\lambda_1+2\lambda_2=0,\ p_2=e^{-(1+\lambda_1+2\lambda_2)}\\
 p_0+p_1+p_2-1=0\\
 p_1+2p_2-\frac{2}{7}=0
\end{gather}

\section{Problem 4}
Because S is an extensive parameter and using the multiplicative property of $\Omega$, when we double the size of the system:
\begin{gather}
 S=f(\Omega)\\
 2S=f(\Omega^2)\\
 2S-S=S\\
 f(\Omega^2)-f(\Omega)=f(\Omega)\\
 f(\Omega^2)=2f(\Omega)
\end{gather}
Therefore f must have the functional form $k\ln{\Omega}$.

\section{Problem 5}
We start with x=1, where $\ln{x}=x-1$. 
Examining the derivatives:
\begin{gather}
 \frac{d}{dx}\ln{x}=\frac{1}{x}\\
 \frac{d}{dx}(x-1)=1
\end{gather}
We can clearly see that x-1 grows faster than $\ln{x}$ for $x > 1$. 
For $x < 1$ we see that $\frac{1}{x}>1$ so that $\ln{x}$ will decrease faster than x - 1 as $x\rightarrow 0$.

\section{Problem 6}
We can write x as $2^{\log_2{x}}$. Taking the natural logarithm of both sides:
\begin{gather}
 \ln{x}=\ln{ 2^{\log_2{x}} }=\log_2{x}\ln{2}\\
 \log_2{x}=\frac{\ln{x}}{\ln{2}}
\end{gather}



\end{document}
