\documentclass[a4paper,12pt]{article}
\usepackage[utf8]{inputenc}
\usepackage{amsmath}
\usepackage{amsfonts}
\usepackage{amssymb}
\usepackage{graphicx}
\usepackage{braket}

\numberwithin{equation}{section}
\renewcommand\thesubsection{\alph{subsection}}
\newcommand{\bvp}[1]{\mathbf{#1}'}
\newcommand{\bv}[1]{\mathbf{#1}}

%opening
\title{Statistical Mechanics I HW3}
\author{Vincent Baker}

\begin{document}

\maketitle

\section{Problem 1}
a. The volume of an N-dimensional hypersphere of radius R is:
\begin{gather}
 V_N(R)=\frac{\pi^{\frac{N}{2}}}{\Gamma(\frac{N}{2}+1)}R^N
\end{gather}
We find the probability of finding a point inside the sphere of radius 0.99999999 by dividing the volume of that sphere by the volume of the unit sphere (R=1).
\begin{gather}
 P_{.99999999}= V_N(.99999999)/V_N(.99999999)\\
 P_{.99999999}= .99999999^N
\end{gather}
b. When N-3 the probability inside the shell is practically 1 $(1-3E^{-8})$.
For $N=N_A$ the probability inside the shell is practically 0 $(\ll 1E^{-10})$.
\\
c. This demonstrates that for the microcanonical ensemble consisting of very large numbers ($\ge N_A$) of molecules $\Sigma (E) \approxeq \Gamma (E)$.
Therefore the entropy can be defined as either $k \ln{\Sigma (E)}$ or $k \ln{\Gamma (E)}$.

\section{Problem 2}
The harmonic oscillator Hamiltonian is a quadratic form that represents an ellipse in phase space.
Writing the Hamiltonian in the common $\frac{x^2}{a^2}+\frac{y^2}{b^2}=1$ form:
\begin{gather}
 H=p^2+(M\omega q)^2=2MU\\
 \frac{p^2}{2MU}+\frac{M\omega^2q^2}{2U}=1\\
 a=\sqrt{2MU},\ b=\sqrt{\frac{2U}{M\omega^2}}
\end{gather}
So we have an ellipse with semimajor axis $\sqrt{2MU}$, semiminor axis $\sqrt{\frac{2U}{M\omega^2}}$, and area $\frac{2\pi U}{\omega}$.
\\
a) With N independent harmonic oscillators the system will consist of N ellipses, each with 2 coordinates, in phase space. 

\section{Problem 3}
a) For the two-level system of N particles the energy U is $n_1E$. 
With $n_1$ and N fixed, $n_0$ is also fixed.
The number of states $\Gamma(U)$ is $\left(\substack{N\\n_1}\right)$.
The entropy is then:
\begin{gather}
 S=k\ln{\frac{N!}{n_0!n_1!}}
\end{gather}
b) From the binomial theorem, the total number of states is $\sum_{k=0}^N \left(\substack{N\\k}\right)=2^N$.
The statistics for $n_0$ and $n_1$ will be calculated in the same manner.
\begin{gather}
 <n_0>=\sum_{k=0}^N \left(\substack{N\\k}\right)k / 2^n\\
 <n_0>=\frac{N2^{N-1}}{2^N}=\frac{N}{2}\\
 <n_0^2>=\sum_{k=0}^N \left(\substack{N\\k}\right)k^2 / 2^n\\
 <n_0^2>=\frac{(N+N^2)2^{N-2}}{2^N}=\frac{N^2+N}{4}
\end{gather}
The mean square fluctuation is:
\begin{gather}
 \frac{<n_0^2>-<n_0>^2}{<n_0>^2}=\frac{1}{N}
\end{gather}
The analysis for $n_1$ is identical.
c) We find the temperature from the entropy.
\begin{gather}
 S(U)=k\ln{\Gamma(U)}\\
 S=k\ln{\frac{N!}{n_0!n_1!}}\\
 \frac{S}{k}=-n_0\ln{\frac{n_0}{N}}-n_1\ln{\frac{n_1}{N}}\\
 \frac{S}{Nk}=-\frac{n_0}{N}\ln{\frac{n_0}{N}}-\frac{n_1}{N}\ln{\frac{n_1}{N}}\\
 \frac{n_0}{N}=\frac{U}{NE},\ \frac{n_1}{N}=1-\frac{U}{NE}
\end{gather}

d) States with negative temperature have higher energy than states with positive temperature. 
Heat will flow from the system with negative temperature to the system with positive temeprature.

\section{Problem 4}
Following Pathria's discussion of the simple harmonic oscillator we take the general coordinate $\theta$ and angular momentum L:
\begin{gather}
 \theta=\frac{h}{\ell}\cos{(\omega t+\phi)}\\
 L=m\ell^2\frac{d\theta}{dt}=-hm\ell\omega\sin{(\omega t+\phi)}
\end{gather}
We can now write the phase space path in the form of an ellipse:
\begin{gather}
 \frac{q^2}{(\frac{h}{\ell})^2}+\frac{p^2}{(hm\ell\omega)^2}=
 \cos^2{(\omega t+\phi)}+\sin^2{(\omega t+\phi)}=1
\end{gather}
With semimajor axis $\frac{h}{\ell}$ and semiminor axis $hm\ell\omega$ the area of the ellipse is $\pi m\omega h^2$.
When p=0 the energy is $E=\frac{1}{2}m\omega^2h^2$ and the period of the pendulum is $\frac{2\pi}{\omega}$.
So we find that $E\tau = \pi m\omega h^2$, the area of the ellipse.

\section{Problem 5}
The uncorrected and corrected entropy formulas are:
\begin{gather}
 S_{uncorrected}=Nk\ln{Vu^{\frac{3}{2}}}+s_0\\
 S_{corrected}=Nk\ln{\frac{V}{N}u^{\frac{3}{2}}}+s_1
\end{gather}
For dissimilar gasses the $N_1$ and $N_2$ cannot be combined into a total N.
Each gas expands to take up the total volume $V=V_1+V_2$.
\begin{gather}
 
\end{gather}


\end{document}
