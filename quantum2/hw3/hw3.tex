\documentclass[a4paper,12pt]{article}
\usepackage[utf8]{inputenc}
\usepackage{amsmath}
\usepackage{amsfonts}
\usepackage{amssymb}
\usepackage{graphicx}
\usepackage{braket}

\numberwithin{equation}{section}
\renewcommand\thesubsection{\alph{subsection}}
\newcommand{\bvp}[1]{\mathbf{#1}'}
\newcommand{\bv}[1]{\mathbf{#1}}

%opening
\title{Quantum II HW3}
\author{Vincent Baker}

\begin{document}

\maketitle

\section{Problem 1}
a. We expand the trace of both AB and BA:
\begin{gather}
 trace(AB)=\sum_i\sum_ja_{ij}b_{ji}\\
 trace(BA)=\sum_i\sum_jb_{ij}a_{ji}
\end{gather}
But i and j are just index variables over the same range, so we can write 1.2 as:
\begin{gather}
 trace(BA)=\sum_j\sum_ib_{ji}a_{ij}
\end{gather}
This is a sum of the same terms as 1.1, so $trace(AB)=trace(BA)$.\\
b. We define the product matrix $BCD...Z\equiv P$. 
Then we have $trace(AP)=trace(PA)$ which we have proved in part a.\\
c. The proof in A holds only for the diagonal elements used in the trace (since they have the same row/column indices).
The general elements of BC and CB will be different, so $trace(ABC)\ne trace(ACB)$.\\
d. The diagonal elements of $\ket{u}\bra{v}$ are $u_iv_i$, so the trace is $\sum u_iv_i=\braket{v|u}$.

\section{Problem 2}
For each matrix, we check its eigenvalues and symmetry to see if it is a valid density matrix. 
If it is a density matrix we determine if it represents a pure state through the eigenvalues (one eigenvalue is 1, others are 0).
If it does represent a pure state, we determine the pure state by finding the eigenvector corresponding to the eigenvalue that equals 1.\\ \\
$\rho_1$ and $\rho_4$ have negative eigenvalues and are therefore not valid desnity matrices.
\\In $\rho_3$, collecting the three ineer products into one constant C, we have:
\begin{gather}
 \rho_3=\frac{C}{3}\ket{u}\bra{v}\\
 \rho_3=\frac{C}{3}\braket{v|u}=0
\end{gather}
So $\rho_3$ is not a valid density matrix.\\
$\rho_5$ is a valid density matrix but does not represent a pure state.\\
$\rho_2$ is a valid density matrix that represents the pure state $\left(\begin{smallmatrix}3/5\\4/5  \end{smallmatrix}\right)$.

\section{Problem 3}
With the two eignvalues 1,-1 the operator of the dynamical variable is $\left(\begin{smallmatrix}1&0\\0&-1\end{smallmatrix}\right)$.
The states in vector form are:
\begin{gather}
 \ket{+}=\begin{bmatrix}1\\0 \end{bmatrix}\\
 \ket{-}=\begin{bmatrix}0\\-1 \end{bmatrix}
\end{gather}
\\
a. For the pure states $\ket{\theta}=\sqrt{\frac{1}{2}}\left(\ket{+}+e^{i\theta}\ket{-}  \right)$ we find:
\begin{gather}
 \bra{\theta}M\ket{\theta}=\bra{\theta}\sqrt{\frac{1}{2}}\begin{bmatrix}\ket{+}\\-e^{i\theta}\ket{-} \end{bmatrix}\\
 \bra{\theta}M\ket{\theta}=\frac{1}{2}\left(\braket{+|+}-(e^{-i\theta}e^{i\theta})\braket{-|-}\right)\\
 \bra{\theta}M\ket{\theta}=\frac{1}{2}(1-1)=0
\end{gather}
For the mixed state $\rho=\frac{1}{2}\left(\ket{+}\braket{+|+|-}\bra{-}  \right)$, writing the inner product as a constant C, we find:
\begin{gather}
 \rho=\frac{C}{2}\left(\ket{+}\bra{-} \right)\\
 \rho^\dagger M \rho=\rho^\dagger \frac{C}{2} \begin{bmatrix}1&0\\0&-1\end{bmatrix} \begin{bmatrix}0&1\\0&0\end{bmatrix}\\
 \rho^\dagger M \rho=\rho^\dagger \frac{C}{2} \begin{bmatrix}1\\0\end{bmatrix}\\
 \rho^\dagger M \rho=\frac{C^2}{4}\ket{+}\bra{-}\ket{+}=\frac{C^2}{4}\ket{+}\left\{ \braket{-|+} \right\}=0
\end{gather}
Since $\ket{+},\ket{-}$ are orthogonal.\\
b. The pure and mixed states can be distinguished by studying their time evolution. 
The pure state evolution is just a phase factor, whereas the mixed state will undergo amplitude transitions.

\section{Problem 4}
To find the probability that $\sigma_y$ is positive we first find its eigenstates.
Solving the inidicial equation we find that the eigenvalues are $\pm\frac{\hbar}{2}$ as expected.
We write the eigenvectors as $e_{-\hbar/2}=\frac{1}{\sqrt{2}}\left( \begin{smallmatrix}i\\1\end{smallmatrix} \right)$
and $e_{\hbar/2}=\frac{1}{\sqrt{2}}\left( \begin{smallmatrix}i\\-1\end{smallmatrix} \right)$.\\
We project the state vector $\left( \begin{smallmatrix}\alpha\\ \beta \end{smallmatrix} \right)$ 
on to the state with energy $\frac{\hbar}{2}$ by:
\begin{gather}
 \frac{1}{\sqrt{2}}\braket{\alpha\ \beta | i\ -1}=\frac{\alpha i-\beta}{\sqrt{2}}
\end{gather}
Since $\frac{\hbar}{2}$ is the only positive state the probahility that $\sigma_y$ is positive is $\frac{\alpha^2+\beta^2}{2}$.

\section{Problem 5}
For the operator M the eigenvalues are $\{-\sqrt{2}, 0, \sqrt{2} \}$.
The eigenvector correspong to e=0 is $e_2=[-\frac{1}{\sqrt{2}}\ 0\ \frac{1}{\sqrt{2}}]$.
We find the conditional probability $P(M=0|p_i)=\bra{e_2}M\ket{e_2}$.\\
\begin{gather}
 P(M=0| \rho_1)=\frac{1}{4}\\
 P(M=0| \rho_2)=0\\
 P(M=0| \rho_3)=\frac{1}{2}
\end{gather}

\section{Problem 6}
For $R=\left( \begin{smallmatrix}6&-2\\-2&9\end{smallmatrix} \right)$ and an arbitrary state vector $\left( \begin{smallmatrix}a\\b\end{smallmatrix} \right)$ we calcualte $\braket{R^2}$ two ways.\\
a. We calculate $\braket{R^2}$ directly.
\begin{gather}
 R^2=\begin{bmatrix}40&-20\\-30&35 \end{bmatrix}\\
 \braket{\Psi|R^2|\Psi}=40a^2-60ab+85b^2
\end{gather}
\\
b. We find the eigenvalues and eigenvectors of R.
\begin{gather}
 R= \begin{bmatrix}6&-2\\-2&9\end{bmatrix}\\
 (6-\lambda)(9-\lambda)-4=0\\
 \lambda^2-15\lambda+50=0\\
 \lambda=\{5,10\}\\
 e_{5}:x_1=2x_2,\ e_5= \frac{1}{\sqrt{5}}\begin{bmatrix}2\\1\end{bmatrix}\\
 e_{10}:x_1=-\frac{1}{2}x_2,\ e_10= \frac{1}{\sqrt{5}}\begin{bmatrix}1\\-2\end{bmatrix}
\end{gather}
We can write $\ket{a\ b}$ as a linear combination of the two eigenvectors by calculating the projection of $\ket{a\ b}$ onto each one.
\begin{gather}
 \braket{a\ b|e_5}=\frac{2a+b}{\sqrt{5}}\\
 \braket{a\ b|e_{10}}=\frac{a-2b}{\sqrt{5}}
\end{gather}
We can now evaluate $\braket{R^2}$.
\begin{gather}
 \braket{R^2}=\braket{a\ b|e_5}^25^2+\braket{a\ b|e_{10}}^210^2\\
 \braket{R^2}=40a^2-60ab+85b^2
\end{gather}
6.12 and 6.2 are the same result as expected.

\end{document}
