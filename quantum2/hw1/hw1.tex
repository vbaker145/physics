\documentclass[a4paper,12pt]{article}
\usepackage[utf8]{inputenc}
\usepackage{amsmath}
\usepackage{amsfonts}
\usepackage{amssymb}
\usepackage{graphicx}
\usepackage{braket}

\numberwithin{equation}{section}
\renewcommand\thesubsection{\alph{subsection}}
\newcommand{\bvp}[1]{\mathbf{#1}'}
\newcommand{\bv}[1]{\mathbf{#1}}

%opening
\title{Quantum II HW1}
\author{Vincent Baker}

\begin{document}

\maketitle

\section{Problem 1}
We write the angular parts of the three wavefunctions as linear combinations of spherical harmonics, using the identity $\sin{\phi}=\frac{e^{i\phi}-e^{-i\phi}}{2i}$.
\begin{gather}
 \psi_1(\theta,\phi)=\sin{\theta}\sin{\phi}=c_1Y_1^1 + c_2Y_1^{-1}\\
 \psi_2(\theta,\phi)=\cos^2{\theta}=c_1Y_2^0 + c_2Y_0^0\\
 \psi_3(\theta,\phi)=\sin{\theta}\cos{\theta}\sin{\phi}=c_1Y_2^1 + c_2Y_2^{-1}
\end{gather}
Solving for the constants and collecting terms we find:
\begin{gather}
 \psi_1(\theta,\phi)=\sin{\theta}\sin{\phi}=i\sqrt{\frac{2\pi}{3}}\left (Y_1^1 + Y_1^{-1}\right)\\
 \psi_2(\theta,\phi)=\cos^2{\theta}=\frac{1}{\sqrt{5\pi}}Y_2^0+\frac{1}{\sqrt{4\pi}}Y_0^0\\
 \psi_3(\theta,\phi)=\sin{\theta}\cos{\theta}\sin{\phi}=i\sqrt{\frac{2\pi}{15}} \left (Y_2^1 + Y_2^{-1}\right)
\end{gather}
Since $L^2\ket{\ell\ m}=\hbar^2\ell(\ell+1)\ket{\ell\ m}$, $L_z\ket{\ell\ m}= \hbar m\ket{\ell\ m}$,
and $\braket{\ell m|\ell m }=1$, we find:
\begin{gather}
 \braket{\Psi_1|L^2|\Psi_1} = i\sqrt{\frac{2\pi}{3}}4\hbar^2,\ \braket{\Psi_1|L_z|\Psi_1}=i\sqrt{\frac{2\pi}{3}}\hbar(1-1)=0\\
 \braket{\Psi_2|L^2|\Psi_2} = \sqrt{\frac{1}{5\pi}}6\hbar^2,\ \braket{\Psi_2|L_z|\Psi_2}=i\sqrt{\frac{1}{4}}\hbar(0+0)=0\\
 \braket{\Psi_3|L^2|\Psi_3} = i\sqrt{\frac{2\pi}{15}}12\hbar^2,\ \braket{\Psi_3|L_z|\Psi_3}=i\sqrt{\frac{2\pi}{15}}\hbar(1-1)=0
\end{gather}


\section{Problem 2}
For J=1. we have $J_+=\left(\begin{smallmatrix}0&\sqrt{2}&0\\0&0&\sqrt{2}\\0&0&0\end{smallmatrix}\right)$,
$J_-=\left(\begin{smallmatrix}0&0&0\\ \sqrt{2}&0&0\\0&\sqrt{2}&0\end{smallmatrix}\right)$, and
$J_z=\left(\begin{smallmatrix}1&0&0\\0&0&0\\0&0&-1\end{smallmatrix}\right)$. Since $J_x=\frac{1}{2}(J_++J_-),J_y=\frac{1}{2i}(J_+-J_-)$,
our matrices and their squares are:
\begin{gather}
 J_x = \frac{1}{2}\begin{bmatrix}0&\sqrt{2}&0\\\sqrt{2}&0&\sqrt{2}\\0&\sqrt{2}&0 \end{bmatrix}
 J_x^2 = \frac{1}{4}\begin{bmatrix}2&0&2\\0&4&0\\2&0&2 \end{bmatrix}\\
 J_y = \frac{1}{2i}\begin{bmatrix}0&\sqrt{2}&0\\-\sqrt{2}&0&\sqrt{2}\\0&-\sqrt{2}&0 \end{bmatrix}
 J_y^2 = \frac{1}{4}\begin{bmatrix}2&0&-2\\0&4&0\\-2&0&2 \end{bmatrix}\\
 J_z = \begin{bmatrix}1&0&0\\0&0&0\\0&0&-1\end{bmatrix}
 J_z^2 = \begin{bmatrix}1&0&0\\0&0&0\\0&0&1\end{bmatrix}
\end{gather}
We put the squared matrices into MATLAB and show directly that they commute (see p2.m). 
The sum of the squared matrices is $2\bv{I_3}$.

\section{Problem 3}
a) To find the probability that the total spin is S, we need the braket of the total state with the two component states 
$\braket{\substack{S\\M}|\substack{s_1\\m_1}\substack{s_2\\m_2}}$. 
We first write the state $\ket{\substack{S\\M}}$ in terms of the component states.
\begin{gather}
 \ket{\substack{S\\M}}=\ket{\substack{s_1\ s_2\\m_1\ m_2}}\braket{\substack{s_1\ s_2\\m_1\ m_2}|\substack{S\\M}}\\
 \bra{\substack{S\\M}}=\braket{\substack{s_1\ s_2\\m_1\ m_2}|\substack{S\\M}}\bra{\substack{s_1\ s_2\\m_1\ m_2}}\\
 \braket{\substack{S\\M}|\substack{s_1\ s_2\\m_1 \ m_2}}=
     \braket{\substack{s_1\ s_2\\m_1\ m_2}|\substack{S\\M}}\braket{\substack{s_1\ s_2\\m_1\ m_2}|\substack{s_1\ s_2\\m_1\ m_2}}\\
 \braket{\substack{S\\M}|\substack{s_1\ s_2\\m_1 \ m_2}}=
     \braket{\substack{s_1\ s_2\\m_1\ m_2}|\substack{S\\M}}    
\end{gather}
b) For an ``unpolarized'' state the expectation value of the total spin is 0 since
\begin{gather}
 <\bv{\sigma}>=Trace(\rho \bv{\sigma} )=\bv{a}
\end{gather}
where $\bv{a}$ is the polarization vector.


\section{Problem 4}
We prove the identity:
\begin{gather}
 (\sigma\cdot A)(\sigma\cdot B)=(A\cdot B)I_2+i\sigma(A\times B)
\end{gather}
The dot products of the Pauli matrices with A and B are:
\begin{gather}
 \sigma\cdot A = \begin{bmatrix}A_z & A_x-iA_y\\ A_x+iA_y & -A_z\end{bmatrix}\\
 \sigma\cdot B = \begin{bmatrix}B_z & B_x-iB_y\\ B_x+iB_y & -B_z\end{bmatrix}
\end{gather}
Multiplying the two matrices and simplifying:
\begin{equation}
\begin{split}
 (\sigma\cdot A)(\sigma\cdot B)=(A\cdot B)I_2+(A_yB_z-A_zB_y)\begin{bmatrix}0 & i\\ i &  0\end{bmatrix}\\
     +(A_zB_x-A_xB_z)\begin{bmatrix}0 & 1\\ -1 &  0\end{bmatrix}\\ 
     +(A_xB_y-A_yB_x)\begin{bmatrix}i & 0\\ 0 & -i\end{bmatrix}
\end{split}
\end{equation}
We reocgnize the terms of the cross product $A\times B$ in the last three terms.
Pulling out a factor of i from all three terms, we have:
\begin{equation}
 \begin{split}
  (\sigma\cdot A)(\sigma\cdot B)=(A\cdot B)I_2+i \{ (A_yB_z-A_zB_y)\begin{bmatrix}0 & 1\\ 1 &  0\end{bmatrix}\\
     +(A_zB_x-A_xB_z)\begin{bmatrix}0 & i\\ -i &  0\end{bmatrix}\\ 
     +(A_xB_y-A_yB_x)\begin{bmatrix}1 & 0\\ 0 & -1\end{bmatrix}\}
 \end{split}
\end{equation}
We have now recovered $\sigma_x,\sigma_y, \sigma_z$ and have proved 3.1.

\section{Problem 5}
Two spin $\frac{1}{2}$ particles interact through the potential $V(r)=V_1(r)+\sigma_1\cdot \sigma_2V_2(r)$.
We will show that the spin-dependent potential can be split into two potentials based on addition of spin.
We start with $\sigma=\frac{2}{\hbar}\bv{S}, \sigma \cdot\sigma=\frac{4}{\hbar^2}S^2$. We take the total spin $\bv{S}=\bv{S_1}+\bv{S_2}$.
\begin{gather}
 S_1+S_2=S\\
 (S_1+S_2)^2=S^2\\
 S_1^2+2S_1S_2+S_2^2=S^2\\
 S_1 \cdot S_2 = \frac{1}{2}\left (S^2-S_1^2-S_2^2 \right )
\end{gather}
Since both particles are spin $\frac{1}{2}$ we have $S_1 \cdot S_1=S_2 \cdot S_2=\frac{3}{4}$.
The values of $m_1=m_2=\pm \frac{1}{2}$, so the value of $M=\{1,0,-1\}$ and therefore $S=1, S \cdot S = \{2,0\}$.
Using 5.4 we find that 
\begin{gather}
S_1 \cdot S_2=\frac{1}{2}\left( \{2, 0\} - \frac{3}{4}-\frac{3}{4} \right)\\
S_1 \cdot S_2=\{\frac{1}{4}, -\frac{3}{4}\}
\end{gather}
With $\sigma \cdot\sigma=\frac{4}{\hbar^2}S^2$ we have therefore shown that $V(r)=V_1(r)+\sigma_1\times \sigma_2V_2(r)$
can be written as two equations:
\begin{gather}
 V(r)=V_1(r)+V_2(r)\\
 V(r)=V_1(r)-3V_2(r)
\end{gather}

\section{Problem 6}
With $J=0$ the system has a single eigenstate $\ket{0 0}$ and is therefore spherically symmetric.
With $J=\frac{1}{2}$ the system has a 2D space defined by eigenstates $\ket{\frac{1}{2}\ \frac{1}{2}},\ \ket{\frac{1}{2}\ -\frac{1}{2}}$.
With a 2D space the system cannot exhibit an electric quadrupole moment, only a dipole moment.

\section{Problem 7}
a) Starting with a Hamiltonian that couples an electric quadrupole moment to the gradient of the electric field:
\begin{gather}
 H_p=C\{ S_iS_j\Phi_{ij}\}
\end{gather}
Transforming to the principal axes coordinate system the cross-derivative terms are zero, so only terms with i=j survive.
\begin{gather}
 H_p=C\{S_x^2\Phi_{xx}+S_y^2\Phi_{yy}+S_z^2\Phi_{zz}\}
\end{gather}
b) We can also write this in the form:
\begin{gather}
 H_p = A(3S_z^2-\bv{S}\cdot\bv{S})+B(S_+^2+S_-^2)\\
 H_p = A(2S_z^2-S_x^2-S_y^2)+B(2S_x^2-2S_y^2)\\
 H_p = S_x^2(2B-A)+S_y^2(-2B-A)+S_z^2(2A)
\end{gather}
Equating coefficients between 5.2 and 5.5 we have:
\begin{gather}
 2B-A=C\Phi_{xx}\\
 -2B-A=C\Phi_{yy}\\
 2A=C\Phi_{zz}
\end{gather}
We find that $A=\frac{C}{2}\Phi_{zz}$, $B=\frac{C}{4}(\Phi_{xx}-\Phi_{yy} )$.\\ \\
c) For a spin $\frac{3}{2}$ system the 4x4 matrix representations of $S^2, S_z, S_+,S_-$ are:
\begin{gather}
 S^2 = \frac{15}{4}\bv{I_4}\\
 S_z=\begin{bmatrix}\frac{3}{2}&0&0&0\\0&\frac{1}{2}&0&0\\0&0&-\frac{1}{2}&0\\0&0&0&-\frac{3}{2} \end{bmatrix}\\
 S_z^2 = \begin{bmatrix}\frac{9}{4}&0&0&0\\0&\frac{1}{4}&0&0\\0&0&\frac{1}{4}&0\\0&0&0&\frac{9}{4} \end{bmatrix}\\
 S_+=\begin{bmatrix}0&\sqrt{3}&0&0\\0&0&2&0\\0&0&0&\sqrt{3}\\0&0&0&0\end{bmatrix}\\
 S_-=S_+^\dagger\\
 S_+^2+S_-^2=\begin{bmatrix}0&0&2\sqrt{3}&0\\0&0&0&2\sqrt{3}\\2\sqrt{3}&0&0&0\\0&2\sqrt{3}&0&0\end{bmatrix}
\end{gather}
The energy eigenvalues are $3A-2\sqrt{3}B, -\frac{11}{4}A-2\sqrt{3}B, -\frac{11}{4}A+2\sqrt{3}B, 3A+2\sqrt{3}B$.



\end{document}
