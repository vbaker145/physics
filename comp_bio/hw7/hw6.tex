\documentclass[a4paper,11pt]{article}
\usepackage[utf8]{inputenc}
\usepackage{amsmath}
\usepackage{amsfonts}
\usepackage{amssymb}
\usepackage{graphicx}
\usepackage{braket}

\numberwithin{equation}{section}
\renewcommand\thesubsection{\alph{subsection}}
\newcommand{\bvp}[1]{\mathbf{#1}'}
\newcommand{\bv}[1]{\mathbf{#1}}
\newcommand{\ez}{\epsilon_0}
\newcommand{\eo}{\epsilon_1}
\newcommand{\lrp}[1]{\left({#1}\right)}
\newcommand{\lrb}[1]{\left\{{#1}\right\}}


%opening
\title{Computational Biophysics HW7}
\author{Vince Baker}

\begin{document}

\maketitle

\section{3.9}
The conventional solution in cylindrical coordinates takes the seperation constant $k^2$, resulting in $\sinh, \cosh$ solutions for the z equation.
In this problem we have boundary conditions $\Phi(\rho,\phi,0)=0$ and $\Phi(\rho,\phi,L)=0$.
These cannot be satisfied by the hyperbolic functions.
Instead, we take the separation constant to be $-k^2$ so that we get the $\sin$ solution for z which will satisfy the boundary conditions.
We are then working with the modified Bessel functions.\\
We are looking for an interior solution, so we discard the Neumann terms. 
The $\phi$ solutions are the typical exponential combinations, with the usual integer restriction on $m$ so they are single-valued.
We can then write the general solution in cylindrical coordinates:
\begin{align}
 \Phi(\rho,\phi,z) &= \sum_{m=0}^\infty \sum_{n=1}^\infty I_m(\frac{n\pi}{L}\rho)\sin{\lrp{\frac{n\pi}{L}z}}\lrp{A_{mn}e^{im\phi}+B_{mn}e^{-im\phi}}
\end{align}
Where we have removed the $n=0$ term, since it will be zero due to the sine function.\\
We now solve for the coefficients using the other provided boundary condition:
\begin{align}
 \Phi(b,\phi,z) &= V(\phi,z)\\
 V(\phi,z) &= \sum_{m=0}^\infty \sum_{n=1}^\infty I_m(\frac{n\pi}{L}b)\sin{\lrp{\frac{n\pi}{L}z}}\lrp{A_{mn}e^{im\phi}+B_{mn}e^{-im\phi}}
\end{align}
We multiply both sides by a $e^{-im'\phi}\sin{\frac{n\pi}{L}z'}$ and integrate, using the delta function relations:
\begin{align}
 \int_0^{2\pi} e^{i(m-m')\phi}\ d\phi &= 2\pi\delta(m-m')\\
 \int_0^L \sin(n\pi z/L)\sin(n'\pi z/L)\ dz &= \frac{L}{2}\delta(n-n')
\end{align}
We then have:
\begin{gather}
 \int_0^{2\pi} \int_0^L V(\phi,z)e^{-im\phi}\sin{\lrp{\frac{n\pi}{L}z}}\ d\phi\ dz =  I_m(\frac{n\pi}{L}b) \pi L A_{mn}\\
 A_{mn} = \lrp{I_m(\frac{n\pi}{L}b) \pi L}^{-1} \int_0^{2\pi} \int_0^L V(\phi,z)e^{-im\phi}\sin{\lrp{\frac{n\pi}{L}z}}\ d\phi\ dz
\end{gather}
Following the same process, but multiplying by $e^{im'\phi}\sin{\lrp{\frac{n\pi}{L}z'}}$, we find the $B_{mn}$:
\begin{gather}
 B_{mn} = \lrp{I_m(\frac{n\pi}{L}b) \pi L}^{-1} \int_0^{2\pi} \int_0^L V(\phi,z)e^{im\phi}\sin{\lrp{\frac{n\pi}{L}z}}\ d\phi\ dz
\end{gather}
Inserting the coefficients into (1) we now have a complete expression for the potential inside the cylinder.


\section{4.1}
a) The charge distribution consists of 4 point charges, all at $r=a$ and $\theta = \pi/2$.
The four $\phi$ angles are $0, \pm\pi/2, \pi$. 
We can write the charge distribution in terms of delta functions:
\begin{align}
 \rho &= q\delta(r-a)\delta(\theta-\pi/2)\lrb{\delta(\phi)+\delta(\phi-\pi/2)-\delta(\phi-\pi/2)-\delta(\phi-\pi)}
\end{align}
The coefficients are then:
\begin{align}
 q_{\ell m} &= \int Y_{\ell m}^*(\theta,\phi)r^\ell \rho(\bv{x}) d^3x\\
 q_{\ell m} &= qa^\ell \lrp{Y_{\ell m}^*(0,0)+Y_{\ell m}^*(0,\pi/2)-Y_{\ell m}^*(0,-\pi/2)-Y_{\ell m}^*(0,\pi)}
\end{align}
We pull out the common $\theta$ part of the spherical harmonics:
\begin{align}
 q_{\ell m} &= qa^\ell P_\ell(0)\sqrt{ \frac{(2\ell+1)(\ell-m)!}{4\pi(\ell+m)!} } \lrp{e^{im0}+e^{im\pi/2}-e^{-im\pi/2}-e^{im\pi}}
\end{align}
When m is even the first and last $\phi$ terms will cancel, as will the second and third. Therefore $q_{\ell m}=0$ for m even.
When m is odd the $\phi$ terms reduce to $2\mp2i$ with the sign alternating.
We can then write the coefficients as:
\begin{align}
 q_{\ell m} &= 2qa^\ell P_\ell(0)\sqrt{ \frac{(2\ell+1)(\ell-m)!}{4\pi(\ell+m)!} } \lrp{1-i^m}\ (\text{m odd})
\end{align}
\\
b) The charge density in b consists of three point charges, so we again write the charge density in terms of delta functions:
\begin{align}
 \rho &= -2q(\delta(r))+q\delta(r)\delta(\theta)+q\delta(r)\delta(\theta-\pi)\\
\end{align}





\section{4.9ab}


\end{document}
