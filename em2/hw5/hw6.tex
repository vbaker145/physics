\documentclass[a4paper,11pt]{article}
\usepackage[utf8]{inputenc}
\usepackage{amsmath}
\usepackage{amsfonts}
\usepackage{amssymb}
\usepackage{graphicx}
\usepackage{braket}

\numberwithin{equation}{section}
\renewcommand\thesubsection{\alph{subsection}}
\newcommand{\bvp}[1]{\mathbf{#1}'}
\newcommand{\bv}[1]{\mathbf{#1}}
\newcommand{\ez}{\epsilon_0}
\newcommand{\eo}{\epsilon_1}
\newcommand{\lrp}[1]{\left({#1}\right)}
\newcommand{\lrb}[1]{\left\{{#1}\right\}}


%opening
\title{Electromagnetic Theory II HW6}
\author{Vince Baker}

\begin{document}

\maketitle

\section*{8.4}
The general wave equation for a cylindrical waveguide is:
\begin{align}
 \lrp{\nabla_t^2+\gamma^2}\psi &= 0\\
 \psi &= E_z\textbf{ (TM) ,}B_z\textbf{ (TE)}
\end{align}
The solutions that are regular at $\rho=0$ are:
\begin{align}
 \psi &= A_m J_m(\gamma\rho)e^{\pm im\phi}
\end{align}
The boundary conditions on the conductor surface $\rho=R$ will be different for TE and TM modes.
For TM modes the E field vanishes at the surface, so the eigenvalues are the zeros of $J_m(\gamma R)$.
For TE modes the normal derivative of the magnetic field vanishes at the surface, so the eigenvalues are the zeros of $J'_m(\gamma R)$.
Therefore the mode frequencies are:
\begin{align}
 \omega_{m,n} &= \frac{Z_m(n)}{\sqrt{\mu\epsilon}R}\textbf{ (TM)}\\
 \omega_{m,n} &= \frac{Z'_m(n)}{\sqrt{\mu\epsilon}R}\textbf{ (TE)}
\end{align}
Where we have defined $Z_m(n)$ as the Nth zero of $J_m$ and $Z'_m(n)$ as the Nth zero of $J'_m$. 
The first few roots are:
\begin{align}
 Z_0 &= 2.41, 5.52, 8.65\\
 Z_1 &= 3.83, 7.02, 10.17\\
 Z_2 &= 5.14, 8.41, 11.62\\
 Z'_0 &= 3.83, 7.02, 10.17\\
 Z'_1 &= 1.84, 5.33, 8.54\\
 Z'_2 &= 3.05, 6.71, 9.97
\end{align}
The lowest root is $Z'_1(1)$, so the the dominant mode is $TE_{11}$. 
Listing the dominant mode and the next four higher modes:
\begin{tabular}{l | c | r}
 Mode & Frequency & Ratio\\
 \hline
 $TE_{11}$ & 1.84  & 1\\
 $TM_{01}$ & 2.41  & 1.31 \\
 $TE_{21}$ & 3.05 & 1.66\\
 $TE_{01},TM_{11}$ & 3.83 & 2.08\\
 $TM_{21}$ & 5.14 & 2.79
\end{tabular}
\\ \\
b) The attenuation coefficients are found from (Jackson 8.57):
\begin{align}
 \beta_\lambda &= -\frac{1}{2P}\frac{dP}{dz}
\end{align}
The power and power loss for TE and TM modes are:
\begin{align}
 P_{TM} &= \frac{1}{2\sqrt{\mu\epsilon}}\lrp{\frac{\omega}{\omega_\lambda}}^2 \lrp{1-\frac{\omega_\lambda^2}{\omega^2}}^{1/2}\epsilon \int_A \psi^* \psi\ da\\
 \frac{dP}{dz}_{TM} &= \frac{1}{2\sigma\delta}\lrp{\frac{\omega}{\omega_\lambda}}^2\oint_C \frac{1}{\mu^2\omega_\lambda^2}|\frac{\partial \psi}{\partial n}|^2\ d\ell\\
 P_{TE} &= \frac{1}{2\sqrt{\mu\epsilon}}\lrp{\frac{\omega}{\omega_\lambda}}^2 \lrp{1-\frac{\omega_\lambda^2}{\omega^2}}^{1/2}\mu \int_A \psi^* \psi\ da\\
 \frac{dP}{dz}_{TM} &= \frac{1}{2\sigma\delta}\lrp{\frac{\omega}{\omega_\lambda}}^2
	      \oint_C \frac{1}{\mu\epsilon\omega_\lambda^2}\lrp{1-\frac{\omega_\lambda^2}{\omega^2}}|\bv{n}\times\nabla_t\psi|^2 
	      +|\frac{\partial \psi}{\partial n}|^2\ d\ell
\end{align}
We can evaluate the power expressions using the orthogonality of the Bessel functions:
\begin{align}
 \int_0^1 xJ_m(xu)J_m(xu)\ dx &= \frac{1}{2}\lrp{J_{m+1}(u)}^2
\end{align}
For the TM modes, integrating from $\rho=0$ to $R$ with $u \equiv Z_m(n)/R$ making the substitution $\rho' = \rho/R$ we find:
\begin{align}
 P_{TM} &= \frac{1}{2}\sqrt{\frac{\epsilon}{\mu}}\lrp{\frac{\omega}{\omega_\lambda}}^2\lrp{1-\frac{\omega_\lambda^2}{\omega^2}}^{1/2} 2\pi\int_0^1R^2\rho' \left[ J_m(\rho' Z_m(n))\right]^2\ d\rho'\\
 P_{TM} &= \frac{1}{2}\sqrt{\frac{\epsilon}{\mu}}\lrp{\frac{\omega}{\omega_\lambda}}^2\lrp{1-\frac{\omega_\lambda^2}{\omega^2}}^{1/2}\pi R^2\left[J_{m+1}(Z_m(n))\right]^2
\end{align}
For the TE modes we are working with the zeros of the derivatives of the Bessel functions. 




\section*{8.6}
We have solved the cylindrical waveguide in the previous problem, the cylindrical cavity frequencies come from the modified expression for $\gamma$ in a cavity:
\begin{align}
 \gamma^2 &= \mu\epsilon\omega^2 - \lrp{\frac{p\pi}{d}}^2\\
 \omega^2_{\lambda p} &= \frac{1}{\mu\epsilon}\lrb{\gamma_\lambda^2+\lrp{\frac{p\pi}{d}}^2}
\end{align}
Where $p=0,1,2..$ for TM modes (cosine solutions in z) and $p=1,2,3...$ for TE modes (sine solutions in z).
Writing the modes in terms of the zeros of the Bessel functions and their derivatives, and pulling out a factor of $1/R^2$:
\begin{align}
 \omega_{m,n,p} &= \frac{1}{\sqrt{\mu\epsilon}R}\sqrt{\lrp{Z_m(n)^2+\lrp{\frac{p\pi R}{d}}^2}}\textbf{ (TM)}\\
 \omega_{m,n,p} &= \frac{1}{\sqrt{\mu\epsilon}R}\sqrt{\lrp{Z'_m(n)^2+\lrp{\frac{p\pi R}{d}}^2}}\textbf{ (TE)}\\
\end{align}



\section*{9.3}

\end{document}
