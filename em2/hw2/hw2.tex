\documentclass[a4paper,11pt]{article}
\usepackage[utf8]{inputenc}
\usepackage{amsmath}
\usepackage{amsfonts}
\usepackage{amssymb}
\usepackage{graphicx}
\usepackage{braket}

\numberwithin{equation}{section}
\renewcommand\thesubsection{\alph{subsection}}
\newcommand{\bvp}[1]{\mathbf{#1}'}
\newcommand{\bv}[1]{\mathbf{#1}}
\newcommand{\ez}{\epsilon_0}
\newcommand{\eo}{\epsilon_1}
\newcommand{\lrp}[1]{\left({#1}\right)}
\newcommand{\lrb}[1]{\left\{{#1}\right\}}


%opening
\title{Electromagnetic Theory II HW2}
\author{Vince Baker}

\begin{document}

\maketitle

\section{6.4}
a) The relevant form of Ohm's law is:
\begin{align}
 \bv{J} = \sigma \bv{E}
\end{align}
We start in the frame that is rotating with the sphere.
In this frame $\bv{J}=0$ and therefore $\bv{E}=0$ since there is no excess charge. 
We can then transform to the lab frame using the relation:
\begin{align}
 \bv{E}_{lab} = \bv{E}_{rot}+\bv{v}\times \bv{B}\\
 \bv{E}_{lab} = \bv{v}\times \bv{B}
\end{align}
We find $\bv{B}$ from the given magnetic moment. 
Since the sphere is uniformly magnetized the magnetization has the same direction and magnitude at each point in the sphere.
The magnetic moment is the volume integral of the magnetization, so it is clear that $M$ is the magnitude of the magnetization.
We choose our coordinates so that the magnetic moment and magnetization are along the z axis.\\
The magnetic field of a uniformly magnetized body can be determined using the magnetic potential formalism.
The result for a sphere is:
\begin{align}
 \bv{B} &= \frac{2}{3}\mu_0 \bv{M} = \frac{2}{3}\mu_0M\hat{z}\ \text{(inside sphere)}
\end{align}
The problem expresses the E field in terms of the magnetic moment $m=\frac{4}{3}\pi R^3M$, so we rewrite the B field:
\begin{align}
 \bv{B} &= \frac{1}{2\pi R^3}\mu_0 m\hat{z}
\end{align}
We now need an expression for $\bv{v} \times \bv{B}$. 
The axis of rotation is aligned with the axis of magnetization, so $\bv{v} = \bv{\omega} \times \bv{r}$ will always be perpendicular to the z axis.
The E field in the lab frame is:
\begin{align}
 \bv{E}_{lab} &= \frac{1}{2\pi R^3}\mu_0 m \omega (\hat{z} \times \bv{r} \times \hat{z})\\
 \bv{E}_{lab} &= \frac{1}{2\pi R^3}\mu_0 m \omega (x\hat{x}+y\hat{y})\\
 \bv{E}_{lab} &= \frac{1}{2\pi R^3}\mu_0 m \omega \rho \hat{\rho}\\
\end{align}
We can find the charge density from $\nabla \cdot \bv{E} = \frac{\rho}{\ez}$.
\begin{align}
 \rho &= \frac{\ez \mu_0 m \omega}{2\pi R^3}\\
 \rho &= \frac{m \omega}{2\pi R^3 c^2}
\end{align}
\\
b) To find the multipole expansion we want the field in spherical coordinates. 
\begin{align}
 \bv{E}_{lab} &= \frac{1}{2\pi R^3}\mu_0 m \omega (x\hat{x}+y\hat{y})\\
 \bv{E}_{lab} &= \frac{1}{2\pi R^3}\mu_0 m \omega (r\sin{\theta}(\sin{\theta}\hat{r}+\cos{\theta}\hat{\theta}))\\
 \bv{E}_{lab} &= \frac{1}{2\pi R^3}\mu_0 m \omega (r\sin^2{\theta}\hat{r}+r\sin{\theta}\cos{\theta}\hat{\theta})
\end{align}
The general solution for the potential outside the sphere, given the axial symmetry, is:
\begin{align}
 \Phi &= \sum_{\ell=0}^\infty A_\ell r^{-(\ell+1)}P_\ell(\cos{\theta})\\
 \bv{E} &= -\nabla \Phi\\
 \bv{E} &= \sum_{\ell=0}^\infty\lrb{ \lrp{A_\ell(\ell+1)r^{-(\ell+2)}P_\ell(\cos{\theta})}\hat{r}+\lrp{-A_\ell r^{-(\ell+2)}\frac{\partial}{\partial \theta}P_\ell(\cos{\theta})}\hat{\theta} }
\end{align}
The tangential electric field at the boundary must be continuous, and the $\theta$ term in the electric field includes $\sin{\theta}\cos{\theta}$.
Taking the derivative of $P_2(\cos{\theta})=\frac{1}{4}(1+3\cos{2\theta})$ we recover the correct form using the double angle formula. 
Solving for the coefficient using the boundary condition:
\begin{align}
 \frac{\mu_0 m \omega R}{2\pi R^3}\sin{\theta}\cos{\theta} &= -A_2 R^{-4}(-3\sin{\theta}\cos{\theta})\\
 A_2 = \frac{\mu_0 m \omega R^2}{6\pi}
\end{align}
Therefore the electric field outside the sphere is:
\begin{align}
 \bv{E} &= \frac{\mu_0 m \omega R^2}{2\pi}\lrp{\frac{1}{r^4}P_2(\cos{\theta})\hat{r}+\frac{1}{r^4}\cos{\theta}\sin{\theta}\hat{\theta}}
\end{align}
We look to express this in a multipole expansion by finding the multipole moments. 
Through orthogonality only the $q_{20}$ moment will survive.
Using equation 4.11 in Jackson:
\begin{align}
 E_r &= \frac{3}{5\ez}q_{20}\sqrt{\frac{5}{4\pi}}\frac{P_2(\cos{\theta})}{r^4}
\end{align}
Therefore,
\begin{align}
 q_{20} &= \frac{\ez\mu_0 m \omega R^2}{2\pi}\frac{5}{3}\sqrt{\frac{4\pi}{5}}\\
 q_{20} &= \frac{m \omega R^2}{c^2}\sqrt{\frac{5}{4\pi}}\frac{2}{3}
\end{align}
The $Q_{33}$ term of the quadropole moment tensor is:
\begin{align}
 Q_{33} &= 2\sqrt{\frac{4\pi}{5}}q_{20} = \frac{m \omega R^2}{c^2}\frac{4}{3}
\end{align}
Since the quadropole tensor is traceless, $Q_{11}=Q_{22}=-\frac{1}{2}Q_{33}$.

c) We have already found the radial electric fields outside and inside the sphere:
\begin{align}
 \bv{E}^{rad}_{in} &= \frac{\mu_0 m \omega}{2\pi R^3}r\sin^2{\theta}\hat{r}\\
 \bv{E}^{rad}_{out} &= \frac{\mu_0 m \omega R^2}{2\pi} \frac{1}{r^4}P_2(\cos{\theta})\hat{r}
\end{align}
Since a dielectric medium is not specified we assume all media have permittivity $\ez$, and the radial continuity equation reduces to:
\begin{align}
 (E^{rad}_{in}-E^{rad}_{out}) &= \frac{\rho}{\ez}\\
 \rho &= \frac{m\omega}{2\pi c^2R^2}\lrp{\sin^2{\theta}-P_2(\cos{\theta})}
\end{align}
Writing $P_2(\cos{\theta})$ as $-\frac{1}{2}+\frac{3}{2}\cos^2{\theta}$ using the double-angle formula we recover the desired charge density:
\begin{align}
 \rho &= \frac{m\omega}{3\pi c^2R^2}\lrp{1-\frac{5}{2}P_2(\cos{\theta})}
\end{align}
\\
d) We integrate along the surface of the sphere from $\theta=0$ to $\theta=\frac{\pi}{2}$.
The tangential electric field at the surface $r=R$ is:
\begin{align}
 \bv{E}^{\theta} &= \frac{\mu_0 m \omega}{2\pi R^2}\cos{\theta}\sin{\theta}\hat{\theta}
\end{align}
The line element is $Rd\theta$, leading to the integral:
\begin{align}
 \int \bv{E} \cdot d\ell &= \int_0^{\pi/2}\frac{\mu_0 m \omega}{2\pi R^2}\cos{\theta}\sin{\theta} R\ d\theta\\
 \int \bv{E} \cdot d\ell &= \frac{\mu_0 m \omega}{2\pi R}\int_0^{\pi/2}\cos{\theta}\sin{\theta} d\theta\\
 \int \bv{E} \cdot d\ell &= \frac{\mu_0 m \omega}{4\pi R}
\end{align}

\section{6.6}
a) The field momentum is:
\begin{align}
 \bv{P_{field}} &= \mu_0\ez\int_V \bv{E} \times \bv{H}\ d^3x
\end{align}
The electric field from the point charge Q can be found from Coulomb's law.
The B field of an N-turn toroid using Amperian loops is $\frac{\mu_0NI}{2\pi a}$ inside, 0 outside, and is directed tangential to the radius from the center of the toroid.
The H field is therefore $\frac{NI}{2\pi a}$.
We can then find the field momentum:
\begin{align}
 \bv{P_{field}} &= \mu_0\ez  \frac{NI}{2\pi a} \frac{1}{4\pi\ez a^2} \int_V  \hat{r} \times \hat{\phi} \ d^3x\\
 \bv{P_{field}} &= \frac{INQA\mu_0}{4\pi a^2}\hat{z}
\end{align}
We check the approximation:
\begin{align}
 \bv{P}_{field} &= \frac{1}{c^2}\bv{E}(0) \times \bv{m}
\end{align}
Where $\bv{E}(0)$ is the first term in a Taylor series expansion of the electric field and the approximation is valid when the change in E field intensity is small over the region containing the current.
Each turn in the toroid has a magnetic moment $IA$, and the E field strength at the center of each turn is $\frac{Q}{4\pi\ez}$, so the momentum in each turn is:
\begin{align}
 \bv{P}_{field} &= \frac{1}{c^2} \frac{Q}{4\pi\ez} IA (\hat{r} \times \hat{\phi})\\
 \bv{P_{field}} &= \frac{IQA\mu_0}{4\pi a^2}\hat{z}
\end{align}
Adding the contribution from N turns the approximation provides the exact result.
b) Plugging in the provided values we find a field momentum of $2.00e^{-12}\ Ns$, an electric field strength of $9.00e^5\ V/m$, and a magnetic induction of $4.00e^{-3}$ Tesla.
For comparison, the momentum of the insect is $1e^{-9}\ Ns$.


\end{document}
