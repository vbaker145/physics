\documentclass[a4paper,11pt]{article}
\usepackage[utf8]{inputenc}
\usepackage{amsmath}
\usepackage{amsfonts}
\usepackage{amssymb}
\usepackage{graphicx}
\usepackage{braket}

\numberwithin{equation}{section}
\renewcommand\thesubsection{\alph{subsection}}
\newcommand{\bvp}[1]{\mathbf{#1}'}
\newcommand{\bv}[1]{\mathbf{#1}}


%opening
\title{Statmech II HW4 problem 8.5}
\author{Vince Baker}

\begin{document}

\maketitle

\section{Huang 8.5}
For a system of N non-interacting harmonic oscillators, the energy levels are given by:
\begin{align}
 E_M &= \frac{N}{2}\hbar \omega + M\hbar \omega
\end{align}
Where M are the number of energy quanta distributed among the harmonic oscillators. 
For Maxwell-Boltzmann statistics we can calculate the degeneracy by considering the M indistinguishable quanta as a row of marks, with $N-1$ ``separators'' to put them into bins.
Each separator may be in any of $M$ positions, so the degeneracy is given by the binomial coefficient:
\begin{align}
 W &= \begin{pmatrix}
       M+N-1 \\
       M
      \end{pmatrix}
\end{align}
Fixing the number of oscillators at N, the grand parition function is therefore:
\begin{align}
 Z_N &= z^Ne^{-\beta\frac{N}{2}\hbar \omega}\sum_{M=0}^\infty \begin{pmatrix}
       M+N-1 \\
       M
      \end{pmatrix} e^{-\beta M\hbar \omega}
\end{align}
\\
For Bose-Einstein statistics the oscillators themselves are also indistinguishable. 
So we can imagine ordering the N harmonic oscillators in a row and filling them sequenctially with energy quanta.
There is then one configuration with all M quanta in the first oscillator, there is one configuration with M-1 quanta in the first oscillator and 1 quanta in the second oscillator, and so on.
The degeneracy is therefore the ``partition'' of M in the number theory sense. As an example, $P(3) = (3, 2+1, 1+1+1) = 3$.
Therefore, the grand partition function for Bose-Einstein statistics is:
\begin{align}
 Z_N &= z^Ne^{-\beta\frac{N}{2}\hbar \omega}\sum_{M=0}^\infty P(M) e^{-\beta M\hbar \omega}
\end{align}



\end{document}
