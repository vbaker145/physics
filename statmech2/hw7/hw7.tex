\documentclass[a4paper,11pt]{article}
\usepackage[utf8]{inputenc}
\usepackage{amsmath}
\usepackage{amsfonts}
\usepackage{amssymb}
\usepackage{graphicx}
\usepackage{braket}

\numberwithin{equation}{section}
\renewcommand\thesubsection{\alph{subsection}}
\newcommand{\bvp}[1]{\mathbf{#1}'}
\newcommand{\bv}[1]{\mathbf{#1}}


%opening
\title{Stat mech II HW7}
\author{Vince Baker}

\begin{document}

\maketitle

\section{Problem 7.6}
For $T>T_c$ we will take $\frac{\partial}{\partial \ln{z}}\frac{\partial \ln{z}}{\partial T}$.
We drop the argument z of the Bose integrals for compactness.
\begin{align}
 \frac{C_V}{Nk} &= \frac{15}{4}\frac{g_{5/2}}{g_{3/2}}-\frac{9}{4}\frac{g_{3/2}}{g_{1/2}}\\
 \frac{1}{Nk}\frac{\partial C_v}{\partial \ln{z}} &= \frac{15}{4}\left(g_{3/2}\frac{1}{g_{3/2}}-g_{5/2}\frac{g_{1/2}}{g_{3/2}^2} \right)
	-\frac{9}{4}\left(g_{1/2}\frac{1}{g_{1/2}}-g_{3/2}\frac{g_{-1/2}}{g_{1/2} ^2} \right)\\
  \frac{1}{Nk}\frac{\partial C_v}{\partial \ln{z}} &= \frac{3}{2}-\frac{15}{4}\frac{g_{5/2}g_{1/2}}{g_{3/2}^2}+\frac{9}{4}\frac{g_{3/2}g_{-1/2}}{g_{1/2}^2}
\end{align}
Using the relation $\frac{1}{z}\left(\frac{\partial z}{\partial T}\right)=\frac{\partial \ln{z}}{\partial T}=\frac{3}{2T}\frac{g_{3/2}}{g_{1/2}}$ we have:
\begin{align}
 \frac{1}{Nk}\left(\frac{\partial C_V}{\partial T} \right) &= \frac{1}{T}\left(\frac{45}{8}\frac{g_{5/2}}{g_{3/2}}-\frac{9}{4}\frac{g_{3/2}}{g_{1/2}}
      -\frac{27}{8}\frac{g_{3/2}^2g_{-1/2}}{g_{1/2}^3} \right)
\end{align}
\\
For $T<T_c$ we use the relation:
\begin{align}
 \frac{C_V}{Nk} &= \frac{15}{4}\zeta\left(\frac{5}{2}\right)\frac{v}{\lambda^3}
\end{align}
With $\frac{1}{\lambda} \propto T^{3/2}$ we can take the derivative with respect to T directly.
Adding a factor of $\frac{1}{T}$ we can keep the expression in terms of $\lambda$:
\begin{align}
 \frac{C_V}{Nk} &= \frac{45}{8}\zeta\left(\frac{5}{2}\right)\frac{v}{T\lambda^3}
\end{align}
We have thus proved the required results.
\\
We now examine the discontinuity at $T=T_c$. 
We take the difference of our two expressions as $T \rightarrow T_c$ from above and below.
Using D.9 to make the approximation $(g_v(z) = \zeta(z)$
\section{Problem 7.10}
We start from the expression for particle density in a Bose system, and the Hamiltonian for an ideal gas in a uniform gravitational field:
\begin{align}
 N = \sum_e \frac{1}{z^{-1}e^{\beta e}-1}\\
 H = \frac{p^2}{2m}+mgz
\end{align}
To convert the sum to an integral we need to calculate the density of energy states $a(e)\ de$.

\section{Problem 7.13}
The number of particles with energy e in a Bose gas is:
\begin{align}
 \sum_e \frac{1}{z^{-1}e^{\beta e}-1}
\end{align}
We convert this to an integral in the usual manner, except the gas is now confined to a 2D space.
So the space integral evaluates to A rather than V, and we divide by $h^2$ for the per-particle area.
Our 2D expression for the density of energy states is therefore:
\begin{align}
 a(e)\ de = \frac{2\pi mA}{h^2} e\ de
\end{align}
And our integral expression is:
\begin{align}
 \frac{N}{A} =\frac{2\pi m}{h^2}\int_0^\infty \frac{e\ de}{z^{-1}e^{\beta e}-1}+\frac{1}{A}\frac{z}{1-z}
\end{align}
Where we have removed the singularity at $e=0$ from the integral.
The term on the left is the number of excited particles $N_e$ while the right hand term is the number of particles in the ground state $N_0$.
The solution to the integral is $g_1(z)$.\\
The critical temperature is defined by the number of particles that can be accomodated in the excited states in the limit as $z \rightarrow 1$, where $g_1(z) \rightarrow \zeta(1)$. 
Although $\zeta(1)$ is an undetermined form, the limits from the left and right are $\pm \infty$.
This implies that the system can accomodate any number of paritcles in the excited states and shows that the system will not undergo Bose-Einstein condensation for any finite temperature.
\end{document}
