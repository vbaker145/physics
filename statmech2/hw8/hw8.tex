\documentclass[a4paper,11pt]{article}
\usepackage[utf8]{inputenc}
\usepackage{amsmath}
\usepackage{amsfonts}
\usepackage{amssymb}
\usepackage{graphicx}
\usepackage{braket}

\numberwithin{equation}{section}
\renewcommand\thesubsection{\alph{subsection}}
\newcommand{\bvp}[1]{\mathbf{#1}'}
\newcommand{\bv}[1]{\mathbf{#1}}


%opening
\title{Stat mech II HW8}
\author{Vince Baker}

\begin{document}

\maketitle

\section{Problem 7.21}
we start with the equation:
\begin{align}
 C_P-C_V &= TV\kappa_T\left(\frac{\partial P}{\partial T} \right)^2\\
 C_P-C_V &= T\left(\frac{\partial V}{\partial P}\right)_T\left(\frac{\partial P}{\partial T} \right)_V^2\\
\end{align}
To find an expression for V we calculate the form of the Helmholtz free energy H.
Since the Debye specific heat $C_V \propto T^3$ at low temperature, U has the form:
\begin{align}
 U \simeq T^4+f(V)
\end{align}
Using the thermodynamic relations:
\begin{align}
 dS &= \frac{1}{T}dU + \frac{P}{T}dV\\
 S &\simeq  VT^3+g(V)\text{ (constant volume)}\\
 -\left(\frac{\partial A}{\partial T} \right)_V &= S\\
 A &\simeq VT^4+g(V)\\
 P &= -\left(\frac{\partial A}{\partial V} \right)_T \simeq T^4+g(V)
\end{align}
Where $f(V),g(V)$ are different arbitrary functions of V only. 
We can now calculate the derivatives of the pressure:
\begin{align}
 \left(\frac{\partial P}{\partial T} \right)_V &\simeq  T^3\\
 \left(\frac{\partial P}{\partial V} \right)_T &\simeq  g(V)
\end{align}
Putting these into equation (2) we find:
\begin{align}
 C_P-C_V &= Tg(V)(T^3)^2 \propto T^7
\end{align}

\section{Problem 7.34}
We consider an n-dimensional Debye system and assume the propagation speed is the same in all directions.
Taking the lattice vibrations as phonons with momentum $p=\hbar\bv{k}$ and $\omega=Ak$ we can find $g(\omega)\ d\omega$:
\begin{align}
 g(k)\ dk &= \frac{L^n}{(2\pi)^n} k^{n-1} dk\\
 g(\omega) d\omega &= \frac{L^n}{(2\pi)^n} \frac{\omega^{n-1}}{A^n}  d\omega
\end{align}
The factor of $\omega^{n-1}$ will carry into the calculation of $\omega_{D}$:
\begin{align}
 \int_0^{\omega_D}g(\omega)d\omega = nN\\
 g(\omega) \propto \frac{1}{\omega_D^n}
\end{align}
The temperature dependence of $C_V$ comes from the 






\end{document}
