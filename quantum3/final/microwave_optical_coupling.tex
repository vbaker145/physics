\documentclass[a4paper,10pt,twocolumn]{article}
\usepackage[utf8]{inputenc}
\usepackage{amsmath}
\usepackage{amsfonts}
\usepackage{amssymb}
\usepackage{graphicx}
\usepackage{braket}
\usepackage{sectsty}
\usepackage{biblatex}

\addbibresource{quantum.bib}
\numberwithin{equation}{section}
\renewcommand\thesubsection{\alph{subsection}}
\newcommand{\bvp}[1]{\mathbf{#1}'}
\newcommand{\bv}[1]{\mathbf{#1}}

\sectionfont{\fontsize{10}{10}\selectfont}

%opening
\title{Mechanical coupling of microwave and optical photons}
\author{Vincent Baker, Drexel University Department of Physics}

\begin{document}


\twocolumn[
\begin{@twocolumnfalse}
 \maketitle
 \begin{abstract}
  Quantum electromagnetic phenomenon are of both theoretical and practical interest. 
  Quantum phenomenon are more readily observable at high energies where individual photons are well localized.
  New methods of coherent coupling between optical and microwave systems hold the promise of extending quantum techniques into the microwave regime. 
 \end{abstract}
\end{@twocolumnfalse}
]
\section{Introduction}
Coupling between optical and microwave modes creates a new set of experimental techniques to explore the principles of quantum dynamics.
The coherent transfer of quantum states may be exploited in applications including quantum-enhanced sensing and quantum computing.
Several similar mechanisms for optical/microwave coupling have been reported recently \cite{nanoCrystal, nanoMR}.\\
We will start by reviewing some of the proposed methods for microwave/optical coupling.
We then sketch the analytical exploration of the nanomechanical resonator from \cite{nanoMR} to demonstrate some important aspects of the approach. 
A general discussion of applications is followed by a discussion of aspects of quantum illumination applied to radar systems.

\section{Optical/Microwave Coupling Methods}
Piezoelectric optomechanical crystal.\\
Mechanical resonator.
\section{Analysis of the Mechanical Resonator}
Hamiltonian.\\
Linearized quantum Langevin equations.\\
Correlation between systems and log-negativity.             
\section{Applications of coherent optical/microwave coupling}
Quanum computing.\\
Quantum illumination.
\section{Discussion of quantum radar}
Quantum-enhanced sensing is a relatively young field that seeks to enhance sensing by using quantum entanglement. 
We discuss the particular application of quantum illumination in the microwave regime applied to radar systems.
Radar systems determine the range, bearing and velocity of an object by emitting a microwave signal and measuring the signal reflections.
Radar waveforms are designed to allow the receiver to correlate the return signal from an object of interest in the presence of thermal noise, interfering signals and signal reflections from other objects.
Quantum illumination can improve this correlation and can theoretically provide better performance than any conventional microwave system. 
\\
Following \cite{qi} we look at the theorectical improvement over a classical system. 
We will then examine several challenges for quantum illumination, including pulsed waveforms and antenna arrays.
\printbibliography
\end{document}
