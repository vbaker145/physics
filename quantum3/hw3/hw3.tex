\documentclass[a4paper,11pt]{article}
\usepackage[utf8]{inputenc}
\usepackage{amsmath}
\usepackage{amsfonts}
\usepackage{amssymb}
\usepackage{graphicx}
\usepackage{braket}

\numberwithin{equation}{section}
\renewcommand\thesubsection{\alph{subsection}}
\newcommand{\bvp}[1]{\mathbf{#1}'}
\newcommand{\bv}[1]{\mathbf{#1}}


%opening
\title{Quantum III HW3}
\author{Vince Baker}

\begin{document}

\maketitle

\section{Problem 1}
We start with the double commutation relation $[[H,e^{ik\cdot r}],e^{ik\cdot r}]$. 
Following the usual proof of the Thomas-Reiche-Kuhn sum rule, we first write this as:
\begin{align}
 [H,e^{ik\cdot r}] &=  [\frac{p^2}{2m}+V(r),e^{ik\cdot r}]\\
 [H,e^{ik\cdot r}] &= \frac{-\hbar^2}{2m}\nabla^2(e^{ik\cdot r})+e^{ik\cdot r}\frac{\hbar^2}{2m}\nabla^2\\
 [[H,e^{ik\cdot r}],e^{ik\cdot r}] &= \frac{i\hbar k^2}{m}[p, r] = \frac{\hbar^2k^2}{m}
\end{align}
Now we expand the double commutator and write the expression for the final state:
\begin{align}
 &\bra{n}[He^{ik\cdot r}-e^{ik\cdot r}H,e^{ik\cdot r}]\ket{n}\\
 &2E_n\bra{n}e^{i2k\cdot r}\ket{n}-2\bra{n}e^{ik\cdot r}He^{ik\cdot r} \ket{n}\\
 \begin{split}
 &\sum_m 2E_n\bra{n}e^{ik\cdot r}\ket{m}\bra{m}e^{ik\cdot r}\ket{n}\\&\ -2\sum_m\bra{n}e^{ik\cdot r}\ket{m}\bra{m}H\ket{m}\bra{m}e^{ik\cdot r}\ket{n}
 \end{split}\\
 &\sum_m 2(E_n-E_m)\bra{n}e^{ik\cdot r}\ket{m}^2
\end{align}
Combining 3 and 7, interchanging m for n, we have shown:
\begin{align}
 \sum_n (E_n-E_m)\bra{n}e^{ik\cdot r}\ket{m}^2 &= \frac{\hbar^2k^2}{2m}
\end{align}

\section{Problem 2}
We first evaluate the commutator $[L_z,P_+]$, with $L_z=-i\hbar x\frac{\partial}{\partial y}+i\hbar y\frac{\partial}{\partial x}$
and $P_+=-i\hbar \frac{\partial}{\partial x}+\hbar \frac{\partial}{\partial y}$.
Opening up the commutator and operating on a test function f:
\begin{align}
 \begin{split}
 [L_z,P_+] = &-\hbar^2x\frac{\partial f}{\partial x \partial y}-i\hbar^2x\frac{\partial^2f}{\partial y^2}+\hbar^2y\frac{\partial^2f}{\partial x^2}+i\hbar^2y\frac{\partial^2f}{\partial x \partial y}\\
  &+\hbar^2\left(\frac{\partial f}{\partial y}+x\frac{\partial^2f}{\partial x \partial y} \right)-\hbar^2y\frac{\partial^2 f}{\partial x^2}+i\hbar^2x\frac{\partial^2 f}{\partial y^2}\\
  &-i\hbar^2\left(\frac{\partial f}{\partial x}+y\frac{\partial^2 f}{\partial x \partial y} \right)
 \end{split}\\
 [L_z,P_+] &= \hbar^2\frac{\partial}{\partial y}-i\hbar^2\frac{\partial}{\partial x} = \hbar P_x
\end{align}

\section{Problem 3}
We use the Born approximation with a spherically symmetric potential:
\begin{align}
 f(\theta) &= \frac{-2m}{\hbar^2 k}\int_0^\infty rV(r)\sin{(kr)} dr\\
 f(\theta) &= \frac{2mV_0}{\hbar^2 k}\int_0^\infty r \sin{(kr)} e^{-r/a}dr\\
 f(\theta) &= \frac{4mV_0}{\hbar^2}\frac{a^3}{\left(k^2a^2+1 \right)^2}\\
 \frac{d\sigma}{d\Omega} &= |f(\theta)|^2 = \left\{\frac{4mV_0}{\hbar^2}\frac{a^3}{\left(k^2a^2+1 \right)^2} \right\}^2
\end{align}
Where we have done the integral through table lookup.

\section{Problem 4}
a. The scattered wavefunctions can be written as linear combinations of the spherical Bessel/Neumann functions. 
For hard-sphere scattering we do not consider the origin so we cannot discard the Neumann part of the solution.
For $\ell=1$:
\begin{align}
 u_{k,1} &= aj_1(kr)+bn_1(kr)\\
 u_{k,1} &= a\left(\frac{\sin{kr}}{(kr)^2}-\frac{\cos{kr}}{kr} \right) + b\left(-\frac{\cos{kr}}{(kr)^2}-\frac{\sin{kr}}{kr} \right)
\end{align}
With the boundary condition that $u_{k1}$ vanishes at $r=r_0$ we can pull out a $\frac{1}{kr}$, incorporate a minus sign into a coefficient, 
and express (2) as:
\begin{align}
 u_{k,1} &= a\left(\frac{\sin{kr}}{kr}-\cos{kr} \right) + b'\left(\frac{\cos{kr}}{kr}+\sin{kr} \right)\\
 u_{k,1} &= a\left(\frac{\sin{kr}}{kr}-\cos{kr}+\frac{b'}{a}\left(\frac{\cos{kr}}{kr}+\sin{kr} \right)\right) \\
 u_{k,1} &= C\left(\frac{\sin{kr}}{kr}-\cos{kr}+a\left(\frac{\cos{kr}}{kr}+\sin{kr} \right) \right)
\end{align}
Where we have defined our coefficients to match the form of the equation given.\\
\\
b. The boundary condition at $r=r_0$ can only be met if the expression in parentheses in (5) is 0 at $r_0$.
\begin{gather}
 \frac{\sin{kr}}{kr}-\cos{kr}+a\left(\frac{\cos{kr}}{kr}+\sin{kr} \right) = 0\\
 a = \frac{\cos{kr_0}-\sin{kr_0}/kr_0}{\cos{kr_0}/kr_0+\sin{kr_0}}\\
 a = \frac{kr_0\cos{kr_0}-\sin{kr_0}}{\cos{kr_0}+kr_0\sin{kr_0}}
\end{gather}
c. Using equation 6.4.62 from Sakurai for $\tan{\delta_\ell}$:
\begin{align}
 \tan{\delta_1}(kr_0) &= \frac{j_1(kr_0)}{n_1(kr_0)}\\
 \tan{\delta_1}(kr_0) &= \frac{\sin{kr_0}-kr_0\cos{kr_0}}{\cos{kr_0}+kr_0\sin{kr_0}}
\end{align}
Comparing (10) and (8) we see that $a=-\tan{\delta_1}(kr_0)$.
\\ \\
d. We use the low-energy expansions of the spherical Bessel functions:
\begin{align}
 j_1(kr) &\simeq \frac{kr}{(3)!!} = \frac{kr}{3}\\
 n_1(kr) &\simeq -\frac{(1)!!}{(kr)^2} = -\frac{1}{(kr^2)}
\end{align}
From (9) we see that $\tan{\delta_1} \propto -(kr_0)^3$. 
Using the small-angle approximation we have proved $\delta_1 \propto (kr_0)^3$, which is much smaller than $\delta_0 \propto kr_0$.

\section{Problem 5}
To prove the optical theorem we will compare the form of the stationary wavefunction in both the plane wave and spherical wave bases,
making use of asymptotic form of the expansion of the incoming wave $e^{ikz}$ in the spherical wave basis.
\begin{align}
 \Psi_{k}(r) &= e^{ikz}+\frac{e^{ikr}}{r}\ \text{ (Plane wave basis)}\\
 e^{ikz} &= \sum_{\ell=0}^\infty i^\ell\sqrt{4\pi(2\ell +1)}j_\ell(kr)Y_\ell^0(\theta)\ \text{(spherical basis)}\\
 e^{ikz} &= \sum_{\ell=0}^\infty i^\ell\sqrt{4\pi(2\ell +1)}\left(\frac{e^{-ikr}e^{i\ell(\pi/2)}-e^{ikr}e^{-i\ell(\pi/2)}}{2ikr} \right)Y_\ell^0(\theta)
\end{align}
For a spherically symmetric potential the effect of scattering in the spherical basis is a phase shift in the radial part of the wavefunction.
The asymptotic form is (multiplying through by the phase shift):
\begin{align}
 U^{diff}(r) &= \sum_{\ell=0}^\infty i^\ell \sqrt{4\pi(2\ell +1)}Y_\ell^0(\theta) \frac{e^{-i(kr+\ell(\pi/2))}-e^{i(kr-\ell(\pi/2)+2\delta_\ell)}}{2ikr}
\end{align}
We can now add and subtract (3) from (4), and will get a form that we can compare to 1.
\begin{align}
 \begin{split}
 U^{diff}(r) = &e^{ikz} +\sum_{\ell=0}^\infty i^\ell \sqrt{4\pi(2\ell +1)}Y_\ell^0(\theta)\\
	       &X\left\{-\frac{e^{ikr}e^{i\ell(\pi/2)}}{2ikr}+\frac{e^{ikr}e^{i\ell(\pi/2)}e^{i2\delta_\ell}}{2ikr} \right\}
 \end{split}\\
 U^{diff}(r) = &e^{ikz} +\sum_{\ell=0}^\infty i^\ell \sqrt{4\pi(2\ell +1)}Y_\ell^0(\theta)\frac{e^{ikr}}{kr}e^{-i\ell(\pi/2)}\left(\frac{1-e^{i2\delta_\ell}}{2i} \right)\\
 U^{diff}(r) &= e^{ikz}+\sum_{\ell=0}^\infty i^\ell\sqrt{4\pi(2\ell+1)}\frac{e^{ikr}}{kr}e^{-i\ell(\pi/2)}e^{i\delta_\ell}\ \sin{\delta_\ell}Y_\ell^0(\theta)\\
 f(\theta,\phi) &= \frac{1}{k}\sum_\ell^\infty \sqrt{4\pi(2\ell+1)}Y_\ell^0(\theta)e^{i\delta_\ell}\sin{\delta_\ell}
\end{align}
Where we have used the fact that $e^{-i\ell(\pi/2)}=(-i)^\ell$ for $\ell=0,1,2...$ in the final step.
We can now compute $\sigma_{\text{total}}$ by taking the integral over all angles.
\begin{align}
 \sigma_{\text{total}} &= \int \mid f(\theta)\mid^2 d\Omega\\
 \begin{split}
 \sigma_{\text{total}} = &\int d\Omega Y_\ell^0 Y_{\ell^\prime}^0 \sum_\ell^\infty \sum_{\ell^\prime}^\infty \sqrt{4\pi(2\ell+1)}\sqrt{4\pi(2\ell^\prime+1)}\\
			  &\times \sin{\delta_\ell}\sin{\delta_{\ell^\prime}}
 \end{split}\\
 \sigma_{\text{total}} &= \frac{4\pi}{k^2}\sum_{\ell=0}^\infty (2\ell+1)\sin^2{\delta_\ell}
\end{align}
Where we have used the orthonormalization of the spherical harmonics $Y_\ell^0Y_{\ell^\prime}^0=\delta_{\ell\ell^\prime} $.\\
To prove the optical theorem, we compute the imaginary part of $f(\theta,\phi)$ at $\theta=0$.
Noting that $Y_\ell^0(1)=1$ for all $\ell$:
\begin{align}
 Im\left(f(\theta,\phi)\right) &= \frac{1}{k}\sum_{\ell=0}^\infty (2\ell+1)\sin{\delta_\ell}\ Im\left(e^{i\delta_\ell}\right)\\
 Im\left(f(\theta,\phi)\right) &= \frac{1}{k}\sum_{\ell=0}^\infty (2\ell+1)\sin^2{\delta_\ell}
\end{align}
Comparing with (10) we have proved the optical theorem:
\begin{align}
 \sigma_{\text{total}} &= \frac{4\pi}{k}\ Im\left(f(\theta,\phi)|_{\theta=0} \right)
\end{align}


\end{document}
