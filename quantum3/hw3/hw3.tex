\documentclass[a4paper,11pt]{article}
\usepackage[utf8]{inputenc}
\usepackage{amsmath}
\usepackage{amsfonts}
\usepackage{amssymb}
\usepackage{graphicx}
\usepackage{braket}

\numberwithin{equation}{section}
\renewcommand\thesubsection{\alph{subsection}}
\newcommand{\bvp}[1]{\mathbf{#1}'}
\newcommand{\bv}[1]{\mathbf{#1}}


%opening
\title{Quantum III HW3}
\author{Vince Baker}

\begin{document}

\maketitle

\section{Problem 1}
We start with the double commutation relation $[[H,e^{ik\cdot r}],e^{ik\cdot r}]$. 
Following the usual proof of the Thomas-Reiche-Kuhn sum rule, we first write this as:
\begin{align}
 [H,e^{ik\cdot r}] &=  [\frac{p^2}{2m}+V(r),e^{ik\cdot r}]\\
 [H,e^{ik\cdot r}] &= \frac{-\hbar^2}{2m}\nabla^2(e^{ik\cdot r})+e^{ik\cdot r}\frac{\hbar^2}{2m}\nabla^2\\
 [[H,e^{ik\cdot r}],e^{ik\cdot r}] &= \frac{i\hbar k^2}{m}[p, r] = \frac{\hbar^2k^2}{m}
\end{align}
Now we expand the double commutator and write the expression for the final state:
\begin{align}
 &\bra{n}[He^{ik\cdot r}-e^{ik\cdot r}H,e^{ik\cdot r}]\ket{n}\\
 &2E_n\bra{n}e^{i2k\cdot r}\ket{n}-2\bra{n}e^{ik\cdot r}He^{ik\cdot r} \ket{n}\\
 \begin{split}
 &\sum_m 2E_n\bra{n}e^{ik\cdot r}\ket{m}\bra{m}e^{ik\cdot r}\ket{n}\\&\ -2\sum_m\bra{n}e^{ik\cdot r}\ket{m}\bra{m}H\ket{m}\bra{m}e^{ik\cdot r}\ket{n}
 \end{split}\\
 &\sum_m 2(E_n-E_m)\bra{n}e^{ik\cdot r}\ket{m}^2
\end{align}
Combining 3 and 7, interchanging m for nn, we have shown:
\begin{align}
 \sum_n (E_n-E_m)\bra{n}e^{ik\cdot r}\ket{m}^2 &= \frac{\hbar^2k^2}{2m}
\end{align}

\section{Problem 2}
We first evaluate the commutator $[L_z,P_+]$, with $L_z=-i\hbar x\frac{\partial}{\partial y}+i\hbar y\frac{\partial}{\partial x}$
and $P_+=-i\hbar \frac{\partial}{\partial x}+\hbar \frac{\partial}{\partial y}$.
Opening up the commutator and operating on a test function f:
\begin{align}
 \begin{split}
 [L_z,P_+] = &-\hbar^2x\frac{\partial f}{\partial x \partial y}-i\hbar^2x\frac{\partial^2f}{\partial y^2}+\hbar^2y\frac{\partial^2f}{\partial x^2}+i\hbar^2y\frac{\partial^2f}{\partial x \partial y}\\
  &+\hbar^2\left(\frac{\partial f}{\partial y}+x\frac{\partial^2f}{\partial x \partial y} \right)-\hbar^2y\frac{\partial^2 f}{\partial x^2}+i\hbar^2x\frac{\partial^2 f}{\partial y^2}\\
  &-i\hbar^2\left(\frac{\partial f}{\partial x}+y\frac{\partial^2 f}{\partial x \partial y} \right)
 \end{split}\\
 [L_z,P_+] &= \hbar^2\frac{\partial}{\partial y}-i\hbar^2\frac{\partial}{\partial x} = \hbar P_x
\end{align}

\section{Problem 3}
We use the Born approximation with a spherically symmetric potential:
\begin{align}
 f(\theta) &= \frac{-2m}{\hbar^2 k}\int_0^\infty rV(r)\sin{(kr)} dr\\
 f(\theta) &= \frac{2mV_0}{\hbar^2 k}\int_0^\infty r \sin{(kr)} e^{-r/a}dr\\
 f(\theta) &= \frac{4mV_0}{\hbar^2}\frac{a^3}{\left(k^2a^2+1 \right)^2}\\
 \frac{d\sigma}{d\Omega} &= |f(\theta)|^2 = \left\{\frac{4mV_0}{\hbar^2}\frac{a^3}{\left(k^2a^2+1 \right)^2} \right\}^2
\end{align}
Where we have done the integral through table lookup.


\end{document}
