\documentclass[a4paper,10pt]{article}
\usepackage[utf8]{inputenc}
\usepackage{amsmath}
\usepackage{graphicx}
\numberwithin{equation}{section}
%opening
\title{Math Phys II HW 3}
\author{Vince Baker}

\begin{document}

\maketitle

\begin{abstract}

\end{abstract}

\section{Problem 1}
We investigate solutions of Laplace' equation in two dimensions.
\begin{gather}
 \nabla ^2 U = 0\\
 U(\rho, \phi)=R(\rho)\Phi(\phi)
\end{gather}
Separating variables and dividing through by $R\Phi$:
\begin{gather}
  \frac{1}{\rho}\frac{d}{d\rho}(r R^{'} )\Phi+\frac{1}{\rho^2}\Phi^{''}R=0\\
  \frac{\rho}{R}(rR^{'})^{'}+\frac{\Phi^{''}}{\Phi}=0
\end{gather}
We use $\Phi(\phi)=e^{im\phi}$ as the angular solution, then:
\begin{gather}
 \frac{\rho}{R}(rR^{'})^{'}=m^2\\
 \rho ^2R^{''}+\rho R^{'}-m^2R=0
\end{gather}
We write down the series solution to this problem around the regular singular point 0.
\begin{gather}
 R=\sum c_i \rho ^{\alpha+i}\\
 R^{'}=\sum (\alpha+i)c_i \rho ^{\alpha+i-1}\\
 R^{''}=\sum (\alpha+i)(\alpha+i-1)c_i\rho ^{\alpha+i-2}\\
 \sum (\alpha+i)(\alpha+i-1)c_i\rho ^{\alpha+i}+\sum (\alpha+i)c_i \rho ^{\alpha+i}-m^2\sum c_i \rho ^{\alpha+i}=0
\end{gather}
Collecting the coefficients of the 0th term and equating them to 0:
\begin{gather}
 [\alpha(\alpha-1)+\alpha-m^2]=0\\
 \alpha=\pm m\\
 R=\rho ^{\pm m}
\end{gather}
We can now write the general solution.
\begin{equation}
 U(\rho, \phi)=\sum (a_{m}\rho^{m}+b_m\rho^{-m})e^{im\phi}
\end{equation}
With boundary condition $U(a,\phi)=U_0cos^2\phi$, we find:
\begin{gather}
  U(a, \phi)=\sum (a_{m}a^{m}+b_ma^{-m})e^{im\phi}=U_0cos^2\phi\\
  (a_{m}a^{m}+b_ma^{-m})=\alpha_m\\
  \alpha_m=\frac{1}{\sqrt{2\pi}}\int_0^{2\pi}U_0cos^2\phi e^{im\phi} d\phi\\
  =\frac{U_0}{2\pi}\int_0^{2\pi}\frac{1}{2}[cos2\phi\ +1]e^{im\phi} d\phi\\
  =\frac{U_0}{2\sqrt{2\pi}}\{-\frac{ie^{im\phi}(mcos2\phi\ -2isin2\phi)}{m^2-4}-\frac{i}{m}e^{im\phi} \}_0^{2\pi}\\
  =0
\end{gather}


\section{Problem 2}
\begin{gather}
 \nabla ^2u=\frac{1}{c^2}\frac{\partial ^2u}{\partial t^2}
\end{gather}
We recognize the time part of the solution as $e^{i\omega t}$. 
The sptial solutions of Equation 2.1 are then the solutions to a Helmholtz equation in three dimensions.
\begin{gather}
 u(r, \theta, \phi, t)=\sum a_{\ell mn}j_{\ell}(cr)P_{\ell}^{m}(cos\theta)e^{im\phi}e^{i\omega t},\ \omega^2=\ell^2+m^2+n^2
\end{gather}
The surface boundary conditions $\frac{\partial u}{\partial r}=0,\ r=R_0$, require that $\frac{d}{dr}j_{\ell}(cR_0)=0$.
We calculate $\frac{d}{dr}j_{\ell}(cR_0)$ for $\ell=0,1,2$.
\begin{gather}
 j_0=\frac{sin x}{x}, j_0^{'}=-\frac{-sin x}{x^2}+\frac{cos x}{x}=0\Rightarrow x=4.49\\
 j_1=\frac{sin x}{x^2}-\frac{cosx}{x}, j_1^{'}=x^2sinx+2xcosx-2sinx=0\Rightarrow x=2.08\\
 j_2=\frac{3sinx}{x^3}-\frac{sinx}{x}-\frac{3cosx}{x^2}, j_2^{'}=-x^3cosx+4x^2sinx+9xcosx-9sinx =0 \Rightarrow x=3.34
\end{gather}
We can now solve for $\omega$.

\section{Problem 3}
\begin{gather}
 \nabla^2n+\lambda n=\frac{1}{\kappa}\frac{\partial n}{\partial t}
\end{gather}
Separating the time and space parts, we find:
\begin{gather}
 T^{'}-\kappa cT=0\\
\end{gather}
With the time solution of the form $e^{\alpha t}$, then $\alpha=\kappa c$.
We separate the space part:
\begin{gather}
 \nabla ^2 \chi + k^2 \chi = 0,\ k^2 \equiv \lambda-c^2
\end{gather}
For spherically symmetric modes $\ell=m=0$, and the solution is:
\begin{gather}
 \chi(r)=j_0(kr)
\end{gather}
Applying the boundary condition $n(R)=0$:
\begin{gather}
 j_0(kR)=\frac{sinx}{x}=0
\end{gather}
The first zero will be at $\pi$, so we find $R_0=\frac{\pi}{k}$.
At the critical point $\alpha=0,k^2=\lambda$. 
So the critical radius $R_0=\frac{\pi}{\sqrt{\lambda}}$. \\ \\
Repeating for a hemisphere we apply axial symmetry (m=0) and find:
\begin{gather}
 \chi(r,\theta )=j_{\ell}(kr)P_{\ell}^{0}(cos \theta )
\end{gather}
We now have a boundary on the upper surface (r=R) and on the bottom of the hemisphere ($\theta=\frac{\pi}{2})$.
For the bottom surface boundary, only $P_{1}^{0}=cos\theta$ satisfiesthe the boundary condition so $\ell=1$.
We can now apply the radial boundary condition:
\begin{gather}
 j_1(kR)=\frac{sinx}{x^2}-\frac{cosx}{x}=0 \Rightarrow x=4.49
\end{gather}



\end{document}
