\documentclass[a4paper,10pt]{article}
\usepackage[utf8]{inputenc}
\usepackage{amsmath}
\usepackage{graphicx}
\numberwithin{equation}{section}
%opening
\title{Math Phys II HW 3}
\author{Vince Baker}

\begin{document}

\maketitle

\begin{abstract}

\end{abstract}

\section{Problem 1}
We investigate solutions of Laplace' equation in two dimensions.
\begin{gather}
 \nabla ^2 U = 0\\
 U(\rho, \phi)=R(\rho)\Phi(\phi)
\end{gather}
Separating variables and dividing through by $R\Phi$:
\begin{gather}
  \frac{1}{\rho}\frac{d}{d\rho}(r R^{'} )\Phi+\frac{1}{\rho^2}\Phi^{''}R=0\\
  \frac{\rho}{R}(rR^{'})^{'}+\frac{\Phi^{''}}{\Phi}=0
\end{gather}
We use $\Phi(\phi)=e^{im\phi}$ as the angular solution, then:
\begin{gather}
 \frac{\rho}{R}(rR^{'})^{'}=m^2\\
 \rho ^2R^{''}+\rho R^{'}-m^2R=0
\end{gather}
We write down the series solution to this problem around the regular singular point 0.
\begin{gather}
 R=\sum c_i \rho ^{\alpha+i}\\
 R^{'}=\sum (\alpha+i)c_i \rho ^{\alpha+i-1}\\
 R^{''}=\sum (\alpha+i)(\alpha+i-1)c_i\rho ^{\alpha+i-2}\\
 \sum (\alpha+i)(\alpha+i-1)c_i\rho ^{\alpha+i}+\sum (\alpha+i)c_i \rho ^{\alpha+i}-m^2\sum c_i \rho ^{\alpha+i}=0
\end{gather}
Collecting the coefficients of the 0th term and equating them to 0:
\begin{gather}
 [\alpha(\alpha-1)+\alpha-m^2]=0\\
 \alpha=\pm m\\
 R=\rho ^{\pm m}
\end{gather}
We can now write the general solution.
\begin{equation}
 U(\rho, \phi)=\sum (a_{m}\rho^{m}+b_m\rho^{-m})e^{im\phi}
\end{equation}
With boundary condition $U(a,\phi)=U_0cos^2\phi$, we now look for the coefficients.
We write the boundary condition as $\frac{U_0}{2}+\frac{U_0cos2 \phi}{2}$ and expand the general $e^{im\phi}$ term into cosines and sines.
\begin{gather}
 U(a, \phi)=\sum (a_{m}a^{m}+b_ma^{-m})(A_mcos\ m\phi+B_msin\ m\phi)
\end{gather}
It is clear from the form of the boundary condition that only the m=0 and m=2 coefficients will survive.
Collecting the various coefficients:
\begin{gather}
 U(a, \phi)=A_0 + A_2cos\ 2\phi = \frac{U_0}{2}+\frac{U_0cos2 \phi}{2}\\
 A_0=A_2=\frac{U_0}{2}
\end{gather}
So the general solution is:
\begin{gather}
 U(\rho, \phi)=\frac{U_0}{2}+\frac{U_0}{2}\rho^2cos\ 2\phi
\end{gather}

\section{Problem 2}
\begin{gather}
 \nabla ^2u=\frac{1}{c^2}\frac{\partial ^2u}{\partial t^2}
\end{gather}
We recognize the time part of the solution as $e^{i\omega t}$. 
The sptial solutions of Equation 2.1 are then the solutions to a Helmholtz equation in three dimensions.
\begin{gather}
 u(r, \theta, \phi)=\sum j_{\ell}(kr)P_{\ell}^{m}(cos\theta)e^{im\phi}
\end{gather}
The surface boundary conditions $\frac{\partial u}{\partial r}=0,\ r=R_0$, require that $\frac{d}{dr}j_{\ell}(cR_0)=0$.
We calculate $\frac{d}{dr}j_{\ell}(cR_0)$ for $\ell=0,1,2$.
\begin{gather}
 j_0=\frac{sin x}{x}, j_0^{'}=-\frac{-sin x}{x^2}+\frac{cos x}{x}=0\Rightarrow x=4.49\\
 j_1=\frac{sin x}{x^2}-\frac{cosx}{x}, j_1^{'}=x^2sinx+2xcosx-2sinx=0\Rightarrow x=2.08\\
 j_2=\frac{3sinx}{x^3}-\frac{sinx}{x}-\frac{3cosx}{x^2}, j_2^{'}=-x^3cosx+4x^2sinx+9xcosx-9sinx =0 \Rightarrow x=3.34
\end{gather}
We can now solve for k and find $\omega=kc$.
\begin{gather}
 \omega_{10}=\frac{4.49}{R}c\\
 \omega_{11}=\frac{2.08}{R}c\\
 \omega_{12}=\frac{3.34}{R}c
\end{gather}


\section{Problem 3}
\begin{gather}
 \nabla^2n+\lambda n=\frac{1}{\kappa}\frac{\partial n}{\partial t}
\end{gather}
Separating the time and space parts, we find:
\begin{gather}
 T^{'}-\kappa cT=0
\end{gather}
With the time solution of the form $e^{\alpha t}$, then $\alpha=\kappa c$.
We separate the space part:
\begin{gather}
 \nabla ^2 \chi + k^2 \chi = 0,\ k^2 \equiv \lambda-c
\end{gather}
For spherically symmetric modes $\ell=m=0$, and the solution is:
\begin{gather}
 \chi(r)=j_0(kr)
\end{gather}
Applying the boundary condition $n(R)=0$:
\begin{gather}
 j_0(kR)=\frac{sinx}{x}=0
\end{gather}
The first zero will be at $\pi$, so we find $R_0=\frac{\pi}{k}$.
At the critical point $\alpha=0,k^2=\lambda$. 
So the critical radius $R_0=\frac{\pi}{\sqrt{\lambda}}$. \\ \\
Repeating for a hemisphere we apply axial symmetry (m=0) and find:
\begin{gather}
 \chi(r,\theta )=j_{\ell}(kr)P_{\ell}^{0}(cos \theta )
\end{gather}
We now have a boundary on the upper surface (r=R) and on the bottom of the hemisphere ($\theta=\frac{\pi}{2})$.
For the bottom surface boundary, only $P_{1}^{0}=cos\theta$ satisfies the the boundary condition so $\ell=1$.
We can now apply the radial boundary condition:
\begin{gather}
 j_1(kR)=\frac{sinx}{x^2}-\frac{cosx}{x}=0 \Rightarrow x=4.49\\
 R_0=\frac{4.49}{\sqrt{\lambda}}
\end{gather}
\\ \\
We can now combine the two critical hemispheres into a single sphere of radius $\frac{4.49}{\lambda}$. 
The critical radius of the sphere is only $\frac{\pi}{\lambda}$, so the sphere is now unstable.
The critical radius for the sphere is $\frac{\pi}{\sqrt{\lambda}}$, so we can express the new radius as $1.43R_0$. 
Using 3.4, we write:
\begin{gather}
 k^2=\lambda-\frac{\alpha}{\kappa}=\\
 k_0=\frac{\pi}{R_0},\ k_1=\frac{\pi}{1.43R_0}=\frac{\sqrt{\lambda}}{1.43}\\
 k_1^2=\frac{\lambda}{1.43^2}=\lambda-\frac{\alpha}{\kappa}\\
 \alpha=\kappa\lambda(0.511)\\
 \tau = \frac{1}{\alpha}=\frac{1.957}{\kappa \lambda}
\end{gather}

\section{Problem 4}
The general solution to the homogenous 2D Helholtz equation is:
\begin{gather}
 u(r,\phi)=\sum_m a_mJ_m(kr)e^{im\phi}
\end{gather}
Applying the boundary condition, we get:
\begin{gather}
 u(R,\phi)=\sum_m a_mJ_m(kR)e^{im\phi}=f(\phi)
\end{gather}
we recognize this as the Fourier expansion of $f(\phi)$, with the Fourier coeeficients $c_m=a_mJ_m(kR)$.
We can now combine the coeeficients into a general coefficient $B_m$ and solve for $B_m$.
\begin{gather}
 B_m=\frac{1}{2\pi}\int_0^{2\pi}f(\phi^{'})e^{im\phi^{'}}d\phi^{'}\\
 a_m=\frac{1}{J_m(kR)}\frac{1}{2\pi}\int_0^{2\pi}f(\phi^{'})e^{im\phi^{'}}d\phi^{'}
\end{gather}
Substituting the $a_m$ back into 4.2 and rearranging:
\begin{gather}
 u(r,\phi,\phi^{'})=\frac{1}{J_m(kR)}\frac{1}{2\pi}\int_0^{2\pi}\{f(\phi^{'})e^{im\phi^{'}}d\phi^{'}\}J_m(kr)e^{im\phi}\\
 =\int_0^{2\pi}\sum_m\frac{J_m(kr)}{J_m(kR)}\frac{1}{2\pi}e^{im\phi}f(\phi^{'})e^{im\phi^{'}}d\phi^{'}\\
 =\int_0^{2\pi}K(r,\phi,\phi^{'}f(\phi^{'})d\phi^{'}\\
 K=\sum_m\frac{J_m(kr)}{J_m(kR)}\frac{1}{2\pi}e^{im\phi}e^{im\phi^{'}}
\end{gather}
\\ \\
To solve for $f(\phi)=cos^2\phi$ we substitute into 4.2.
\begin{gather}
 u(R,\phi)=\sum_m a_mJ_m(kR)e^{im\phi}=cos^2\phi
\end{gather}
Using the same approach as problem 1, we find that only the m=0 and m=2 coefficients survive and are equal to $\frac{1}{2J_0(kR)}$ and $\frac{1}{2J_2(kR)}$.
\begin{gather}
 u(r,\phi)=\frac{1}{2J_0(kR)}J_0(kr)+\frac{1}{2J_2(kR)}J_2(kr)cos\ 2\phi
\end{gather}


\end{document}
