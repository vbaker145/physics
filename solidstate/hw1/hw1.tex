\documentclass[a4paper,11pt]{article}
\usepackage[utf8]{inputenc}
\usepackage{amsmath}
\usepackage{amsfonts}
\usepackage{amssymb}
\usepackage{graphicx}
\usepackage{braket}

\numberwithin{equation}{section}
\renewcommand\thesubsection{\alph{subsection}}
\newcommand{\bvp}[1]{\mathbf{#1}'}
\newcommand{\bv}[1]{\mathbf{#1}}
\newcommand{\ez}{\epsilon_0}
\newcommand{\eo}{\epsilon_1}
\newcommand{\lrp}[1]{\left({#1}\right)}
\newcommand{\lrb}[1]{\left\{{#1}\right\}}


%opening
\title{Solid State 1 HW1}
\author{Vince Baker, Jesse Unger}

\begin{document}
\maketitle

\section*{1}
We use the distance convention relative to the distance between the centers of adjacent cubes. 
The first four terms of the Madleung constant are then:
\begin{align}
 T1: \frac{8}{\sqrt{3/4}} &= 9.24\\
 T2: -6\\
 T3: \frac{-12}{\sqrt{2}} &= -8.48\\
 T4: \frac{24}{\sqrt{11/4}} &= 14.47
\end{align}

Alternatively, using the nearest-neighbor distance as 1:
\begin{align}
 T1: 8\\
 T2: -\frac{6}{\sqrt{2}} &= -4.24\\
 T3: -\frac{12}{2} &= -6\\
 T4: \frac{24}{\sqrt{11/2}} &= 10.23
\end{align}
The series is converging.

\section*{2}
We use the Lennard-Jones potential with parameters $\epsilon=50, \sigma=2.74$.
The total potential energy is given by Kittel 3.11:
\begin{align}
 U_{tot} &= 2N\epsilon\lrb{\sum_j\lrp{\frac{\sigma}{p_{ij}R}}^{12}-\sum_j\lrp{\frac{\sigma}{p_{ij}R}}^6 }
\end{align}
For the fcc and bcc crystals the constants are the same, and only the structure-dependent sums are different. 
For the fcc crystal:
\begin{align}
 \sum_j p_{ij}^{-12} &\equiv p_{ij}^{-12} = 12.13\\
 \sum_j p_{ij}^{-6} &\equiv p_{ij}^{-6} = 14.45
\end{align}
For the bcc crystal:
\begin{align}
 \sum_j p_{ij}^{-12} &\equiv p_{ij}^{-12} = 9.11\\
 \sum_j p_{ij}^{-6} &\equiv p_{ij}^{-6} = 12.25
\end{align}
The cohesive energy depends on the lattice distance R. 
The equilibrium value of R can be evaluated from the derivative of the potential energy:
\begin{align}
 \frac{dU}{dR} &= 2N\epsilon\lrb{-\sigma^{12}\lrp{p_{ij}}^{-12}12R^{-12}+\sigma^{6}\lrp{p_{ij}}^{-6}6R^{-7}} = 0\\
 R_0 &= \sigma\lrb{\frac{2p_{ij}^{-12}}{p_{ij}^{-6}}}^{1/6}
\end{align}
The equilibrium value $R0$ is 1.0902 for the fcc structure and 1.0684 for the bcc structure. 
We can now calculate the cohesive energies of the Ne structures:
\begin{align}
 U_{fcc} &= -4.30341(2N\epsilon)\\
 U_{bcc} &= -4.1181(2N\epsilon)\\
 \frac{U_{bcc}}{U_{fcc}} &= 0.9569
\end{align}

\section*{3}  

\section*{4}
The cohesive energy per molecule of an ionic crystal (ignoring the repulsive term) is:
\begin{align}
 U_{tot} &= -\frac{aq^2}{R}
\end{align}
For both Ba+O- and Ba++O-- the internuclear distance is 2.76\AA{} and the Madelung constant is 1.75.
For Ba+O- $q=e$, while for Ba++O-- $q=2e$. 
Calculating the energies in SI and then converting to eV:
\begin{align}
 U_{Ba+O-} &= -\frac{1}{4\pi\ez}\frac{ae^2}{R_0} = 1.46e^{-18} J = -9.12 eV\\
 U_{Ba++O--} &= -\frac{1}{4\pi\ez}\frac{a(2e)^2}{R_0} = 5.84e^{-18} J = -36.48 eV
\end{align}
A configuration will be stable if the cohesive energy plus the electron affinity of the negative ion are greater than the ionization potential of the positive ion.
For Ba+O-, the configuration energy is ($-9.12-1.5+5.19=-5.43 eV$) so the configuration is stable.
For Ba++O-- the configuration energy is ($-36.48+9+9.96=-17.52 eV$) so it is stable as well.
Since the lowest energy configuration is Ba++O-- we expect that to be the valence configuration found in the BaO crystal.

\end{document}
