\documentclass[a4paper,11pt]{article}
\usepackage[utf8]{inputenc}
\usepackage{amsmath}
\usepackage{amsfonts}
\usepackage{amssymb}
\usepackage{graphicx}
\usepackage{braket}

\numberwithin{equation}{section}
\renewcommand\thesubsection{\alph{subsection}}
\newcommand{\bvp}[1]{\mathbf{#1}'}
\newcommand{\bv}[1]{\mathbf{#1}}
\newcommand{\ez}{\epsilon_0}
\newcommand{\eo}{\epsilon_1}
\newcommand{\lrp}[1]{\left({#1}\right)}
\newcommand{\lrb}[1]{\left\{{#1}\right\}}


%opening
\title{Solid State 1 HW3}
\author{Vince Baker, Cooper Voigt}

\begin{document}
\maketitle

\section*{Q1}
1) The primitive lattice vectors of graphite are:
\begin{align}
 \bv{a_1} &= a(\sqrt{3}/2, -1/2, 0)\\
 \bv{a_2} &= a(\sqrt{3}/2, 1/2, 0)\\
 \bv{a_3} &= c(0, 0, 1)
\end{align}
To find the reciprocal lattice we first calculate the cross products:
\begin{align}
 \bv{a_2} \times \bv{a_3} &= (ac/2, -\sqrt{3}ac/2, 0)\\
 \bv{a_3} \times \bv{a_1} &= (ac/2, \sqrt{3}ac/2, 0)\\
 \bv{a_1} \times \bv{a_2} &= (0, 0, \sqrt{3}a^2/2)
\end{align}
We can now calculate the reciprocal lattice vectors:
\begin{align}
 \bv{b_1} &= \frac{2\pi}{a}(1/\sqrt{3},-1 , 0)\\
 \bv{b_2} &= \frac{2\pi}{a}(1/\sqrt{3}, 1 , 0)\\
 \bv{b_3} &= \frac{2\pi}{c}(0, 0 , 1)\\
\end{align}
\\
2) We first need to express the atomic positions in terms of the primitive lattice vectors:
\begin{align}
 (0,0,0) &= 0\bv{a_1}+0\bv{a_2}+0\bv{a_3}\\
 (0,0,c/2) &= 0\bv{a_1}+0\bv{a_2}+1/2\bv{a_3}\\
 a(1/2\sqrt{3},1/2,0) &= -1/3\bv{a_1}+2/3\bv{a_2}+0\bv{a_3}\\
  (-a/2\sqrt{3},-a/2,c/2) &= 1/3\bv{a_1}-2/3\bv{a_2}+1/2\bv{a_3}
\end{align}
We then use the definition of the structure factor (Kittel 2.46) and the common atomic structure factor $f_c$:
\begin{align}
 S_g &= \sum_j f_je^{-i2\pi(hx_j+ky_j+\ell z_j)}\\
 S_g &= f_c\lrb{ 1 + e^{-i\pi\ell}+e^{-i2\pi(-1/3h+2/3k)}+e^{-i2\pi(1/3h-2/3k+1/2\ell)} }
\end{align}
An alternate structure is given in D. D. L. Chung's ``Review Graphite'' in Journal of Materials Science Vol 37 (2002).
The reciprocal vector is expressed as:
\begin{align}
 G_m &= \lrp{\frac{2\pi}{\sqrt{3}a}(m_1+m_2),\frac{2\pi}{a}(m_1-m_2), \frac{2\pi}{c}m_3}
\end{align}
In this form the structure factor is then:
\begin{align}
 S_g &= f_c\lrb{ 1 + e^{-i\pi\ell}+e^{-i(8\pi/3)h}+e^{-i(\pi/3)(3\ell-4h+2k} }
\end{align}
\\
3) Examining the terms in the structure factor we note that when $h=2k$ we have $-1/3h+2/3k=0$ and $1/3h-2/3k=0$.
Under this condition the structure factor simplifies to:
\begin{align}
 S_g &= f_j\lrb{ 1 + e^{-i\pi\ell}+1+e^{-i\pi\ell} }\{h=2k\}
\end{align}
So the forbidden reflections are $h=2k,\ell=1,3,5...$.
\\
Using Chung's formulation we find that ALL planes with $\ell=1,3,5...$ are forbidden.
\\ \\
4) From the given relation the interplanar distance is $d=\frac{c}{\sqrt{\lrp{4c^2/3a^2+1}}}=0.203$.
\\ \\
5) Using Bragg's law with $n=1$, the angle is $\sin^-1{(0.246)}=0.249$ or 14.3 degrees.
\section*{Q2}
1) For a given incident vector $\bv{k}$ there are multiple reflecting planes for which different scattered vectors $\bv{k'}$ satisfy the Bragg condition.
\\ \\
2) The incoming electron beam is oriented paralell to $\bv{b_3}$ so we will not see reflections from reciprocal lattice vectors along that orientation.
\\ \\
3) From the expression for the interplanar spacing, the $(1,0,0)$ reflection has:
\begin{align}
 1/d^2 &= \frac{4}{3}\lrp{\frac{1}{a^2}}
\end{align}
However, the following planes have the same interplanar distance (although different angles of reflection):\\
$(0,1,0),(-1,0,0),(0,-1,0),(1,-1,0),(-1,1,0)$.\\
So we would expect to see reflections from all six of these planes at the same angle in six different directions.
\\
4) Broadening of lines may be caused by largely-constructive interference at angles close to Bragg angles, imperfect collimation of the electron beam, imperfections in the bulk crystal structure, momentum (and hence wavelength) spread among the electrons, and the point spread function of the detector.
The FWHM of the peak is also known to depend on the sample thickness $\tau$ according the the Scherr equation:
\begin{align}
 FWHM &= k\lambda/\tau \cos{(\theta)}
\end{align}


\end{document}
