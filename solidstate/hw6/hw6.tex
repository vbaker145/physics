\documentclass[a4paper,11pt]{article}
\usepackage[utf8]{inputenc}
\usepackage{amsmath}
\usepackage{amsfonts}
\usepackage{amssymb}
\usepackage{graphicx}
\usepackage{braket}

\numberwithin{equation}{section}
\renewcommand\thesubsection{\alph{subsection}}
\newcommand{\bvp}[1]{\mathbf{#1}'}
\newcommand{\bv}[1]{\mathbf{#1}}
\newcommand{\ez}{\epsilon_0}
\newcommand{\eo}{\epsilon_1}
\newcommand{\lrp}[1]{\left({#1}\right)}
\newcommand{\lrb}[1]{\left\{{#1}\right\}}


%opening
\title{Solid State 1 HW6}
\author{Geoffrey Xiao, Joe Tumulty, Vincent Baker}

\begin{document}
\maketitle

\section*{Q1}
Starting with the general expression for U, and using $f(e)=1 (e<e_f), =0(e>e_F)$ at $T=0$:
\begin{align}
 U &= \int_0^\infty eD(e)f(e)de \\
 U &= \int_0^{e_F} \frac{V}{2\pi^2}\lrp{\frac{2m}{\hbar}}^{3/2}e^{3/2}de\\
 U &= \frac{V}{2\pi^2}\lrp{\frac{2m}{\hbar}}^{3/2}\int_0^{e_F}e^{3/2}de \\
 U &= \frac{V}{5\pi^2}\lrp{\frac{2m}{\hbar}}^{3/2}e_F^{5/2}
\end{align}
At $T=0$ we also have $N=\frac{V}{3\pi^2}\lrp{\frac{2m}{\hbar}}^{3/2}e_F^{3/2}$, leading to:
\begin{align}
 U &= \frac{3}{5}Ne_F
\end{align}

\section*{Q2}
For a two-dimensional electron gas the number of allowed states in a square of length L is:
\begin{align}
  \frac{2\pi k^2}{\lrp{2\pi/L}^2} &= \frac{A}{2}k^2 = N
\end{align}
 We then compute the density of states from N:
\begin{align}
 k^2 &= \frac{2m}{\hbar^2}\epsilon\\
 N &= \frac{mA}{\hbar^2}\epsilon\\
 D(\epsilon) &= \frac{dN}{d\epsilon} = \frac{mA}{\hbar^2}
\end{align}
We see that the density of states is constant.

\section*{Q3}


\section*{Q4}
a) The nanometric wire is similar to a waveguide, supporting standing wave modes across the wire and propagating modes along it.
Since the standing wave modes have no physical momentum the total energy minus the standing wave energy must be the kinetic energy:
\begin{align}
 E(n,N) - \epsilon(n) &= \frac{1}{2} mv^2
\end{align}
Solving $\frac{\hbar^2}{2m}\Delta^2 \Psi = E\Psi$ with the given wavefunction we find that:
\begin{align}
 E(N,n) &= \frac{\hbar^2}{2m}\lrb{\lrp{\frac{n\pi}{L_{xy}}}^2+(2\pi N)^2}
\end{align}
Where v is the average electron velocity along the wire, N is the quantum number of the propagating mode, and n is the quantum number ($n_x+n_y$) of the standing wave modes.
Using the expression for kinetic energy
\begin{align}
 E(N,n) - e(n) &= \frac{\hbar^2}{2m}\lrp{2\pi N}^2 = \frac{1}{2}mv^2\\
 N &= \frac{mv}{2\pi\hbar}
\end{align}
Finally, substituting into the expression for $D_N(E)$:
\begin{align}
 D_N(E) &= 2/AN\\
 \frac{\lrp{2\pi\hbar}^2}{2m} &\equiv A\\
 D_N(E) &= \frac{1}{\pi \hbar v}
\end{align}
b) The problem directs us to sum over all n of which $E\ge e_n$. 
The given expression using a step function so that any term with $E-e_n < 0$ is equal to zero.
This is exactly the summation requested.

\end{document}
