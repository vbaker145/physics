\documentclass[a4paper,11pt]{article}
\usepackage[utf8]{inputenc}
\usepackage{amsmath}
\usepackage{amsfonts}
\usepackage{amssymb}
\usepackage{graphicx}
\usepackage{braket}

\numberwithin{equation}{section}
\renewcommand\thesubsection{\alph{subsection}}
\newcommand{\bvp}[1]{\mathbf{#1}'}
\newcommand{\bv}[1]{\mathbf{#1}}
\newcommand{\ez}{\epsilon_0}
\newcommand{\eo}{\epsilon_1}
\newcommand{\lrp}[1]{\left({#1}\right)}
\newcommand{\lrb}[1]{\left\{{#1}\right\}}


%opening
\title{Solid State 1 HW7}
\author{Jesse Unger, Paul Xhori, Vincent Baker}

\begin{document}
\maketitle

\section*{7.1}
a) The energy of a free electron gas in two and three dimensions is:
\begin{align}
 E_{2D} &= \frac{\hbar^2}{2m}\lrp{k_x^2+k_y^2}\\
 E_{3D} &= \frac{\hbar^2}{2m}\lrp{k_x^2+k_y^2+k_z^2}
\end{align}
The corner point at $k_y=k_x=\pi/a$ then has twice the energy of the k vector halfway along the side at $k_x=\pi/a, k_y=0$.\\
b) The corner point at $k_z=k_y=k_x=\pi/a$ has three times the energy of the k vector halfway along the side at $k_x=\pi/a, k_z=k_y=0$.\\
c)
\section*{7.3}
a) The Kronig-Penney model is:
\begin{align}
 (P/Ka)\sin{Ka} + \cos{Ka} &= \cos{ka}
\end{align}
We are looking for the lowest energy band when $k=0$ with $P \ll 1$.
Assuming the first term is small due to the condition on P, and noting that $\varliminf_{x\rightarrow 0}\sin{x}/x = 1$, we have:
\begin{align}
 K &\simeq 0,k=0\\
 \cos{Ka} &\simeq 1-P\\
 K &\simeq \frac{1}{a} \cos^{-1}{(1-P)}
\end{align}
The energy of the lowest band at $k=0$ is $\frac{\hbar^2}{2ma^2}\lrp{\cos^{-1}{(1-P)}}^2$.\\
b) At $k=\pi/a$ we have:
\begin{align}
 \frac{P}{Ka}\sin{Ka}+\cos{Ka} &= -1
\end{align}
To find the band gap we need to find the points at which the function is equal to -1.
We note that the sinc function $\sin{x}/x$ has zeros at $x=\pm n\pi$ with n an integer, so one of the points is at $Ka=\pi$.
Expaning both the sinc function and $\cos{x}$ about $x=\pi$ we have:
\begin{align}
 P(1-x/\pi)-1+\frac{1}{2}\lrp{x-\pi}^2 &= -1\\
 P(1-x/\pi)+\frac{1}{2}\lrp{x-\pi}^2 &= 0
\end{align}
We don't find a closed-form expression from this, but plugging in $P=0.1$ we find the second solution at $x=3.205, K=3.205/a$.
This gives a band gap of $\frac{\hbar^2}{2ma^2}(3.205^2-\pi^2)$.
 
\section*{Q3}
Examining the band structure of graphene in figure 2 of Electron and Phonon Properties of Graphene we see that the top valence band ($\pi $) and the conduction band ($\pi*$) meet at the $K$ point.
This crossing of the valence and conduction bands at a specific point explains the high conductivity of graphene.


\end{document}
