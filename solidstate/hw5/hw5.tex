\documentclass[a4paper,11pt]{article}
\usepackage[utf8]{inputenc}
\usepackage{amsmath}
\usepackage{amsfonts}
\usepackage{amssymb}
\usepackage{graphicx}
\usepackage{braket}

\numberwithin{equation}{section}
\renewcommand\thesubsection{\alph{subsection}}
\newcommand{\bvp}[1]{\mathbf{#1}'}
\newcommand{\bv}[1]{\mathbf{#1}}
\newcommand{\ez}{\epsilon_0}
\newcommand{\eo}{\epsilon_1}
\newcommand{\lrp}[1]{\left({#1}\right)}
\newcommand{\lrb}[1]{\left\{{#1}\right\}}


%opening
\title{Solid State 1 HW5}
\author{Tony Wang, Vincent Baker}

\begin{document}
\maketitle

\section*{Q1}
The atomic motion is restricted to two dimensions. 
With one phonon mode per $(\pi/L)^2$ in K space:
\begin{align}
 N &= (\pi K^2)/(2\pi/L)^2\\
 N &= AK^2/4\pi\\
 \frac{dN}{dK} &= AK/2\pi\\ 
 D(\omega) &=  \frac{dN}{dK}\frac{dK}{d\omega} = A\omega/2\pi v^2
\end{align}
Where in the last step we used the relation $\omega=vK$.\\
We now have an expression for the internal energy under the Debye assumptions:
\begin{align}
 U &= 3\int_0^{\omega_D} d\omega \frac{A\omega}{2\pi v^2}\frac{\hbar \omega}{e^{\beta \hbar \omega}-1}\\
 \omega_D &= \lrp{\frac{4\pi Nv^2}{A}}^{1/2}\\
 \frac{\partial U}{\partial T}_v &= \frac{3A\hbar^2}{2\pi v^2kT^2} \int_0^{\omega_D} d\omega \frac{\omega^3e^{\beta\hbar\omega}}{\lrp{e^{\beta \hbar \omega}-1}^2}
\end{align}
Introducing the Debye temperature $\theta = \frac{\hbar v}{k}\lrp{4\pi N/A}^{1/2}$ and the variable $x=\hbar \omega /kT$ we can write the heat capacity as:
\begin{align}
 C_v &=  \frac{3Ak^3T^2}{2\pi v^2\hbar^2}\int_0^{x_D} dx \frac{x^3e^x}{\lrp{e^x-1}^2}\\
 C_v &=  6kA\lrp{\frac{T}{\theta}}^2\int_0^{x_D} dx \frac{x^3e^x}{\lrp{e^x-1}^2}
\end{align}
For low temperatures $x_d \rightarrow \infty$ and the integral is a constant (7.21 per Wolfram Alpha). 
The heat capacity therefore has a $T^2$ dependence for low temperatures.\\ \\
If adjacent layers are weakly bound compared to the intra-layer interactions then at extremely low temperatures only the low-energy inter-layer phonon modes will be excited.
This essentially becomes a one-dimensional problem, and therefore $C_v \propto T$ since the number of excited phonon modes is proportional to T in one dimension.
\\ 
\section*{Q2}
We first calculate the dilation from Kittel 3.35:
\begin{align}
 \delta &= \frac{\Delta V}{V} \\
 U &= \frac{1}{2}B\delta^2 \approx kT
\end{align}
For $T=300K, B=70e^9 erg/cm$ we find $\delta = 1.09e^{-12} cm^3$.
To find the linear thermal displacement $\Delta a/a$ we take the cube root, $^3\sqrt{\delta} = 1.03e^-9 cm$.

\section*{Q3}
Graphite is made of layers of covalently-bonded graphene, with weak van der Waals forces holding the layers together. 
The thermal conductivity is proportional to the square of the speed of sound in the material, which is just the slope of the linear part of the dispersion relation.
Per our previous investigation the slope is proportional to the square root of the bond strength.
The carbon covalent bond strength is 328 kJ/mol while van der Waals interactions are on the order of 1-10 kJ/mol (per table 3-4 of Kittel), explaining the difference in thermal conductivity.
\end{document}
