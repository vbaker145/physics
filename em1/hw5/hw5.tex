\documentclass[a4paper,11pt]{article}
\usepackage[utf8]{inputenc}
\usepackage{amsmath}
\usepackage{amsfonts}
\usepackage{amssymb}
\usepackage{graphicx}
\usepackage{braket}

\numberwithin{equation}{section}
\renewcommand\thesubsection{\alph{subsection}}
\newcommand{\bvp}[1]{\mathbf{#1}'}
\newcommand{\bv}[1]{\mathbf{#1}}
\newcommand{\ez}{\epsilon_0}
\newcommand{\lrp}[1]{\left({#1}\right)}
\newcommand{\lrb}[1]{\left\{{#1}\right\}}


%opening
\title{Electromagnetic Theory HW4}
\author{Vince Baker}

\begin{document}

\maketitle

\section{Problem 3.1}
The spherical geometry with azimuthal symmetry will have a solution of the form:
\begin{align}
 \Phi(r,\theta) &= \sum_{\ell=0}^\infty (A_\ell r^\ell+B_\ell r^{-(\ell+1)})P_\ell \cos{\theta}
\end{align}
Since we are working the region between two concentric spheres we keep both radial terms.
We'll apply the boundary conditions on the inner and outer sphers and use both equations to determine $A_\ell$ and $B_\ell$.
Starting with the inner sphere boundary, we have:
\begin{align}
 A_\ell a^\ell + B_\ell a^{-(\ell+1)} &= V\frac{2\ell+1}{2} \lrb{\int_0^{\pi/2} P_\ell(\cos{\theta})\sin\theta\ d\theta}
\end{align}
For the outer sphere boundary we find:
\begin{align}
 A_\ell b^\ell + B_\ell b^{-(\ell+1)} &= V\frac{2\ell+1}{2} \lrb{\int_{\pi/2}^{\pi} P_\ell(\cos{\theta})\sin\theta\ d\theta}
\end{align}
Since $\sin\theta$ is an odd function the non-zero even terms in the series vanish, since the non-zero even Legendre polynomials are even functions.
To calculate the integrals we use the result:
\begin{align}
 P'_{n+1}(x)-P'_{n-1}(x) &= (2n+1)P_n(x)\\
 \frac{1}{2n+1}\frac{d}{dx}(P'_{n+1}(x)-P'_{n-1}(x)) &= P_n(x)
\end{align}
Substituing $x\rightarrow\cos\theta$ we can now integrate by parts:
\begin{align}
 \int_0^{\pi/2} P_\ell(\cos{\theta})\sin\theta\ d\theta &= -\int_1^0 P_\ell(x) dx\\
 \int_0^{\pi/2} P_\ell(\cos{\theta})\sin\theta\ d\theta &= \int_0^1 \frac{1}{2\ell+1}\frac{d}{dx}(P'_{\ell+1}(x)-P'_{\ell-1}(x)) dx\\
 \int_0^{\pi/2} P_\ell(\cos{\theta})\sin\theta\ d\theta &= \frac{1}{2\ell+1}\lrb{P_{\ell+1}(x)-P_{\ell-1}(x)}_0^1\\
 \int_0^{\pi/2} P_\ell(\cos{\theta})\sin\theta\ d\theta &= \frac{1}{2\ell+1}(P_{\ell-1}(0)-P_{\ell+1}(0))
\end{align}
We note that the integral in (3) will have the same result but with opposite sign. 
We can now write our equations for the coefficients:
\begin{align}
 A_\ell a^\ell + B_\ell a^{-(\ell+1)} &= \frac{V}{2} (P_{\ell-1}(0)-P_{\ell+1}(0))\\
 A_\ell b^\ell + B_\ell b^{-(\ell+1)} &= -\frac{V}{2} (P_{\ell-1}(0)-P_{\ell+1}(0))
\end{align}
After a lot of algebra, we find the coefficients:
\begin{align}
 A_\ell &= \lrb{\frac{1}{a^\ell}+\frac{1}{a^{2\ell+1}} \frac{a^\ell+b^\ell}{a^\ell b^{-(\ell+1)}-1/a}}
	    \frac{V}{2} (P_{\ell-1}(0)-P_{\ell+1}(0))\\
 B_\ell &= -\lrb{ \frac{a^\ell+b^\ell}{a^\ell b^{-(\ell+1)}-1/a} } \frac{V}{2} (P_{\ell-1}(0)-P_{\ell+1}(0))
\end{align}
We have now found the complete expression for the potential inside the two spheres.

\section{Problem 3.6}
This problem specifies two point charges rather than a boundary. 
Instead of the usual series solution to a boundary value problem, we expand the potentials of the two point charges in terms of spherical harmonics (Jackson 3.70).
Recognizing the azimuthal symmetry we keep only the $m=0$ terms, with the normalization of the spherical harmonics cancelling the $\frac{1}{2\ell+1}$ term:
\begin{align}
 \frac{1}{|\bv{x}-\bvp{x}|} &= 4\pi \sum_\ell \frac{r_<^\ell}{r_>^{\ell+1}}P_\ell(\cos\theta')P_\ell(\cos\theta)
\end{align}
With the charges at $\theta'=0,\pi$, we can then write the complete potential:
\begin{align}
 \Phi(r,\theta) &= \frac{q}{4\pi\ez}\sum_\ell \frac{r_<^\ell}{r_>^{\ell+1}} P_\ell(\cos\theta)\lrb{P_\ell(1)-P_\ell(-1)}
\end{align}
$P_\ell(x)$ is an odd function for $\ell$ odd and an even function for $\ell$ even. 
We then write the potential:
\begin{align}
 \Phi(r,\theta) &= \frac{q}{2\pi\ez}\sum_{\ell=1,3...} \frac{r_<^\ell}{r_>^{\ell+1}} P_\ell(\cos\theta)
\end{align}
\\
b) As $a\rightarrow 0$, $r_<=a=0$ and $r_>=r$. Therefore only the $\ell=0$ term would normally survive.
Keeping the ratio $qa=p/2$ constant so that the potential doesn't vanish, we can pull out an a:
\begin{align}
 \Phi(r,\theta) &= \frac{p}{4\pi\ez} \sum_{\ell=1,3...} \frac{a^{\ell-1}}{r^{\ell+1}} P_\ell(\cos\theta)\\
 \Phi(r,\theta) &= \frac{p}{4\pi\ez} \frac{\cos\theta}{r^2}
\end{align}
c) Since the total potential on the ground sphere is 0, the charge distribution on the sphere must cancel the dipole potential.
Solving the dipole potential at the sphere surface:
\begin{align}
 \Phi(b,\theta) &= \frac{p}{4\pi\ez} \frac{\cos\theta}{b^2} + \Phi_{sphere} = 0\\
 \Phi_{sphere} &= -\frac{p}{4\pi\ez} \frac{\cos\theta}{b^2} 
\end{align}
We examine the potential due to the charge distribution on the sphere.
Clearly only the $\ell=1$ term survives, and the coefficient $A_1=-\frac{p}{4\pi\ez} \frac{\cos\theta}{b^3}$ to match the expression at $r=b$. 
We then combine the two potentials:
\begin{align}
 \Phi(r,\theta) &= \frac{p}{4\pi\ez}\cos\theta\lrb{\frac{1}{r^2}-\frac{r}{b^3}} 
\end{align}
We note that hhis potential vanishes at the sphere surface as required.
\\

\section{Problem 3.14}
We first solve for the line charge density:
\begin{align}
 C \int_{-d}^d(d^2-z^2)dz &= Q\\
 C &= Q\lrp{2d^3-\frac{2}{3}d^3}^{-1}\\
 C &= \frac{3Q}{4d^3}\\
 \lambda(z) &= \frac{3Q}{4d^3}(d^2-z^2),\ |z|<d
\end{align}
This problem includes charge in a region with a spherical boundary, so we use the spherical Green's function.
The charge distribution has azimuthal symmetry so we only keep terms with $m=0$, and can write the Green's function in terms of Legendre polynomials:
\begin{align}
 G(\bv{x},\bvp{x}) &= \sum_{\ell=0}^\infty P_\ell(\cos\theta')P_\ell(\cos\theta)r_<^\ell\lrp{\frac{1}{r_>^{\ell+1}}-\frac{r_>^\ell}{b^{2\ell+1}}}
\end{align}
Since the potential is 0 on the surface of the sphere we only need the volume integral term in the solution:
\begin{align}
 \Phi(\bv{x}) &= \frac{1}{4\pi\ez}\int_V dV\ \rho(\bvp{x})G(\bv{x},\bvp{x})
\end{align}
The volume is 0 everywhere except along the z axis from $-d$ to $d$. 
To use the Green's function in spherical coordinates we break the integral into two pieces along $\theta'=0$ and $\theta'=\pi$.
We then have:
\begin{align}
 \begin{split}
 \Phi(\bv{x}) = &\frac{3Q}{16\pi\ez d^3}\times\\
		&\int_0^d dr'\ (d^2-r'^2) \sum_{\ell=0}^\infty P_\ell(\cos\theta)r_<^\ell\lrp{\frac{1}{r_>^{\ell+1}}-\frac{r_>^\ell}{b^{2\ell+1}}}+\\
                &\int_{-d}^0 dr'\ (d^2-r'^2) \sum_{\ell=0}^\infty (-1)^\ell P_\ell(\cos\theta)r_<^\ell\lrp{\frac{1}{r_>^{\ell+1}}-\frac{r_>^\ell}{b^{2\ell+1}}}\\
 \end{split}
\end{align}
The two integrals are essentially identical except for the order of integration and the factor of $(-1)^\ell)$ in the second integral.
It is clear that if we exchance the sum and integration that all odd terms in the sum will be 0. 
The expression then simplifies to:
\begin{align}
 \begin{split}
 \Phi(\bv{x}) = &\frac{3Q}{8\pi\ez d^3}\times\\
		&\sum_{\ell=0,2...}^\infty \int_0^d dr'\ (d^2-r'^2) P_\ell(\cos\theta)r_<^\ell\lrp{\frac{1}{r_>^{\ell+1}}-\frac{r_>^\ell}{b^{2\ell+1}}}\\
 \end{split}
\end{align}
b) To find the charge density on the sphere we use $\sigma = -\ez\left[\frac{\partial \Phi(n,\theta)}{\partial r}\right]_{r=b}$.
We first evaluate the potential at $r>r'$ since we are looking at the surface of the sphere:
\begin{align}
  \Phi(r,\theta) &= \frac{3Q}{8\pi\ez d^3}\sum_{\ell=0,2...}^\infty \int_0^d dr'\ (d^2-r'^2) P_\ell(\cos\theta)r'^\ell\lrp{\frac{1}{r^{\ell+1}}-\frac{r^\ell}{b^{2\ell+1}}}\\
  \Phi(r,\theta) &= \frac{3Q}{8\pi\ez d^3}\sum_{\ell=0,2...}^\infty P_\ell(\cos\theta)\lrp{\frac{1}{r^{\ell+1}}-\frac{r^\ell}{b^{2\ell+1}}} \int_0^d dr'\ (d^2-r'^2) r'^\ell\\
  \Phi(r,\theta) &= \frac{3Q}{4\pi\ez}\sum_{\ell=0,2...}^\infty P_\ell(\cos\theta)\lrp{\frac{1}{r^{\ell+1}}-\frac{r^\ell}{b^{2\ell+1}}} \frac{d^\ell}{(\ell+1)(\ell+3)}
\end{align}
We now take the derivative with respect to r, evaluated at $r=b$:
\begin{align}
 \frac{\partial \Phi(r,\theta)}{\partial r} &= \frac{3Q}{4\pi\ez}\sum_{\ell=0,2...}^\infty P_\ell(\cos\theta) \frac{d^\ell}{(\ell+1)(\ell+3)}\lrp{-\frac{\ell+1}{r^{\ell+2}}-\frac{\ell r^{\ell-1}}{b^{2\ell+1}}}\\
 \frac{\partial \Phi(b,\theta)}{\partial r} &= -\frac{3Q}{4\pi\ez}\sum_{\ell=0,2...}^\infty P_\ell(\cos\theta) \frac{(2\ell+1)d^\ell}{(\ell+1)(\ell+3)b^{\ell+2}}
\end{align}
With the normal pointing in from the surface we pick up a negative sign, so the final expression for the charge density is:
\begin{align}
 \sigma &= -\frac{3Q}{4\pi}\sum_{\ell=0,2...}^\infty P_\ell(\cos\theta) \frac{(2\ell+1)d^\ell}{(\ell+1)(\ell+3)b^{\ell+2}}
\end{align}
c) With $d \ll b$ the ratio $d^\ell/b^{\ell+2}$ approaches 0 for all terms other than $\ell=0$. 
Therefore, we can approximate the surface charge density by:
\begin{align}
 \sigma &= -\frac{Q}{4\pi b^2}
\end{align}
This is just the total charge on the line distribution spread evenly over the surface of the sphere, so in this limit the problem resembles a point charge at the center of a conducting sphere.



\end{document}
