\documentclass[a4paper,11pt]{article}
\usepackage[utf8]{inputenc}
\usepackage{amsmath}
\usepackage{amsfonts}
\usepackage{amssymb}
\usepackage{graphicx}
\usepackage{braket}

\numberwithin{equation}{section}
\renewcommand\thesubsection{\alph{subsection}}
\newcommand{\bvp}[1]{\mathbf{#1}'}
\newcommand{\bv}[1]{\mathbf{#1}}
\newcommand{\ez}{\epsilon_0}
\newcommand{\lrp}[1]{\left({#1}\right)}
\newcommand{\lrb}[1]{\left\{{#1}\right\}}


%opening
\title{Electromagnetic Theory HW4}
\author{Vince Baker}

\begin{document}

\maketitle

\section{Problem 2.13}
a) For a long cylinder with surface potentials independent of z the solution should have no z-dependence.
The standard separation of Laplace's equation in polar coordinates produces the general series solution:
\begin{align}
 \begin{split}
  \Phi(\rho, \phi) = &A_0 + B\ln{\rho}+\sum_{n=1}^\infty \rho^n\lrp{A_n\cos{n\phi}+B_n\sin{n\phi}}\\
		    &+ \sum_{n=-1}^{-\infty} \rho^n\lrp{C_n\cos{n\phi}+D_n\sin{n\phi}}
 \end{split}
\end{align}
We need a solution that is regular at the origin and single-valued, so the negative series and the logarithmic term coefficients must be 0.
\begin{align}
  \Phi(\rho, \phi) = &A_0 +\sum_{n=1}^\infty \rho^n\lrp{A_n\cos{n\phi}+B_n\sin{n\phi}}
\end{align}
We first solve for the $A_0$ coefficient using the provided boundary conditions on the cylinder inner surface.
\begin{align}
 \Phi(b,\phi) &= \begin{cases}
                 V_1,& -\pi/2\le\phi\le\pi/2 \\
                 V_2,& \pi/2\le\phi\le3\pi/2 \\
                \end{cases}\\
 \int_{-\pi}^\pi \Phi(b,\phi) d\phi &= \int_{-\pi}^\pi A_0 d\phi + \int_{-\pi}^{\pi}\sum_{n=1}^\infty \rho^n\lrp{A_n\cos{n\phi}+B_n\sin{n\phi}}\\
 \pi (V_2+V_1) &= 2\pi A_0 + 0\\
 A_0 &= \frac{V_2+V_1}{2}
\end{align}



\section{Problem 2.15}

\end{document}
