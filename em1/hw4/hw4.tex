\documentclass[a4paper,11pt]{article}
\usepackage[utf8]{inputenc}
\usepackage{amsmath}
\usepackage{amsfonts}
\usepackage{amssymb}
\usepackage{graphicx}
\usepackage{braket}

\numberwithin{equation}{section}
\renewcommand\thesubsection{\alph{subsection}}
\newcommand{\bvp}[1]{\mathbf{#1}'}
\newcommand{\bv}[1]{\mathbf{#1}}
\newcommand{\ez}{\epsilon_0}
\newcommand{\lrp}[1]{\left({#1}\right)}
\newcommand{\lrb}[1]{\left\{{#1}\right\}}


%opening
\title{Electromagnetic Theory HW4}
\author{Vince Baker}

\begin{document}

\maketitle

\section{Problem 2.13}
a) For a long cylinder with surface potentials independent of z the solution should have no z-dependence.
The standard separation of Laplace's equation in polar coordinates produces the general series solution:
\begin{align}
 \begin{split}
  \Phi(\rho, \phi) = &A_0 + B\ln{\rho}+\sum_{n=1}^\infty \rho^n\lrp{A_n\cos{n\phi}+B_n\sin{n\phi}}\\
		    &+ \sum_{n=-1}^{-\infty} \rho^n\lrp{C_n\cos{n\phi}+D_n\sin{n\phi}}
 \end{split}
\end{align}
We need a solution that is regular at the origin and single-valued, so the negative series and the logarithmic term coefficients must be 0.
\begin{align}
  \Phi(\rho, \phi) = &A_0 +\sum_{n=1}^\infty \rho^n\lrp{A_n\cos{n\phi}+B_n\sin{n\phi}}
\end{align}
We first solve for the $A_0$ coefficient using the provided boundary conditions on the cylinder inner surface.
\begin{align}
 \Phi(b,\phi) &= \begin{cases}
                 V_1,& -\pi/2\le\phi\le\pi/2 \\
                 V_2,& \pi/2\le\phi\le3\pi/2 \\
                \end{cases}\\
 \int_{-\pi}^\pi \Phi(b,\phi) d\phi &= \int_{-\pi}^\pi A_0 d\phi + \int_{-\pi}^{\pi}\sum_{n=1}^\infty \rho^n\lrp{A_n\cos{n\phi}+B_n\sin{n\phi}}\\
 \pi (V_2+V_1) &= 2\pi A_0 + 0\\
 A_0 &= \frac{V_2+V_1}{2}
\end{align}
To solve for the $A_n$ and $B_n$ we note the following delta function relations::
\begin{align}
 \int_0^{2\pi}\sin{(n\phi)}\sin{(m\phi)}\ d\phi = \pi\delta_{n,m}\\
 \int_0^{2\pi}\cos{(n\phi)}\cos{(m\phi)}\ d\phi = \pi\delta_{n,m}
\end{align}
Multiplying through by $\cos{(m\phi)}$ and integrating we get the following relations for $A_n$:
\begin{align}
 \int \Phi(b,\phi)\cos{(n\phi)} &= \pi b^n A_n\\
 \int_{-\pi/2}^{\pi/2}V_1\cos{(n\phi)} +&\int_{\pi/2}^{3\pi/2}V_2\cos{(n\phi)}= \pi b^n A_n\\
 A_n &= \frac{1}{\pi b^n n}\lrb{V_1\lrp{\sin{(n\phi)}}_{-\pi/2}^{\pi/2}+V_2\lrp{\sin{(n\phi)}}_{\pi/2}^{3\pi/2}}\\
 A_n &= \frac{2}{\pi b^n n}(-1)^{(n+1)/2}\lrb{V_2-V_1}\text{ (n odd)}
\end{align}
For $B_n$:
\begin{align}
 \int \Phi(b,\phi)\sin{(n\phi)} &= \pi b^n B_n\\
 \int_{-\pi/2}^{\pi/2}V_1\sin{(n\phi)}+&\int_{\pi/2}^{3\pi/2}V_2\sin{(n\phi)} = \pi b^n B_n\\
 B_n &= \frac{-1}{\pi b^n n}\lrb{V_1\lrp{\cos{(n\phi)}}_{-\pi/2}^{\pi/2}+V_2\lrp{\cos{(n\phi)}}_{\pi/2}^{3\pi/2}}\\
 B_n &= 0
\end{align}
\\
We now have a complete expression for the potential:
\begin{align}
 \begin{split}
  \Phi(\rho,\phi) = &\frac{V_1+V_2}{2}+\frac{2(V_2-V_1)}{\pi}\\
                    &\times\sum_{n=1,3,5...}^\infty\frac{\rho^n}{nb^n}\lrb{(-1)^{(n+1)/2}\cos{(n\phi)}}
 \end{split}
\end{align}
We make use the following series expansion of the arctan function:
\begin{align}
 \tan^{-1}(x) &= \sum_{n=0}^\infty \frac{(-1)^2}{2n+1}x^{2n+1}
\end{align}
We recognize that the sum over odd terms can be converted to a regular series by making the substitution $n\rightarrow 2n+1$.
We put the cosine term into an exponential form by writing $\cos{((2n+1)\phi)}= \Re\lrb{e^{i(2n+1)\phi}}$.
We can then write the series as:
\begin{align}
 \begin{split}
  \Phi(\rho,\phi) = &\frac{V_1+V_2}{2}+\frac{(V_1-V_2)}{\pi}\\
                    &\times\sum_{n=0}^\infty\frac{2\rho^{2n+1}}{(2n+1)b^{2n+1}}\lrb{(-1)^n\Re\lrb{e^{i(2n+1)\phi}}}
 \end{split}\\
 \Phi(\rho,\phi) = &\frac{V_1+V_2}{2}+\frac{2(V_2-V_1)}{\pi}\Re\lrb{\tan^{-1}\lrp{\frac{\rho}{b}e^{i\phi}}}
\end{align}
We now use the identity:
\begin{align}
 \Re\lrb{\tan^{-1}(z)} &= \frac{1}{2}\tan^{-1}\lrp{\frac{2\Re(z)}{1-|z|^2}}
\end{align}
We can now simplify the series:
\begin{align}
 \Phi(\rho,\phi) = &\frac{V_1+V_2}{2}+\frac{(V_2-V_1)}{\pi}\tan^{-1}\lrp{\frac{2(\rho/b)\cos{\phi}}{1-\rho^2/b^2}}\\
 \Phi(\rho,\phi) = &\frac{V_1+V_2}{2}+\frac{(V_2-V_1)}{\pi}\tan^{-1}\lrp{\frac{2\rho b \cos{\phi}}{b^2-\rho^2}}
\end{align}
\\
b) To find the surface charge density we take the normal derivative of the potential.
\begin{align}
 \sigma &= \ez\frac{\partial \Phi}{\partial \rho}\\
 \sigma &= \ez\lrp{\frac{V_1-V_2}{\pi}}\frac{1}{1+\lrp{\frac{2b\rho\cos{\phi}}{b^2-\rho^2}}^2}\lrp{2b\cos{\phi}\frac{b^2+\rho^2}{(b^2-\rho^2)^2}}\\
 \sigma &= \ez\lrp{\frac{V_1-V_2}{\pi}}\frac{2b\cos{\phi}(b^2+\rho^2)}{(b^2-\rho^2)^2+(2b\rho\cos{\phi})^2}
\end{align}
Where we have used an expression for the derivative of the arctangent that is valid between $-\pi/2<\phi<\pi/2$.
However, since the second half of the sphere is between $\pi/2<\phi<3\pi/2$ the derivative expression applies to both halves, since the arctan function is identical on both intervals.
Evaluating the expression at $\rho=b$:
\begin{align}
 \sigma &= \ez\lrp{\frac{V_1-V_2}{\pi}}\frac{4b^3\cos{\phi}}{(2b^2\cos{\phi})^2}\\
 \sigma &= \ez\lrp{\frac{V_1-V_2}{\pi}}\frac{1}{b\cos{\phi}}
\end{align}

\section{Problem 2.15}
a) To show that the Green function has an expansion of the form specified, we use the relation:
\begin{align}
 \nabla'^2(G_D(\bv{x},\bvp{x}) &= -4\pi\delta(\bv{x}-\bvp{x})
\end{align}
Taking the second derivative of the suggested expansion:
\begin{align}
 \begin{split}
  \nabla'^2(G_D) = 2\sum_{n=0}^\infty &g_n(y,y')\nabla^2(\sin{(n\pi x)}\sin{(n\pi x')}\\
			    &+\sin{(n\pi x)}\sin{(n\pi x')}\nabla^2(g_n(y,y')
 \end{split}\\
 \begin{split}
 \nabla'^2(G_D) = 2\sum_{n=0}^\infty &\lrp{\sin{(n\pi x)}\sin{(n\pi x')}}\\
				     &\times\lrp{\nabla^2(g_n(y,y')-n^2\pi^2g_n(y,y')}
 \end{split}
\end{align}
Inserting the provided condition on $g_n(y,y')$ and using the same delta-function sine series as the previous problem we see  that the provided expansion satisfies the required relation.
In addition, for Dirichlet boundary conditions, the Green function must vanish on the boundary. The $\sin{n\pi x}$ expressions naturally vanish at $\lrb{0,1}$.
The additional constraints that $g_n(y,y')=0$ at $y'=0,y'=1$ then guarantee the remainder of the boundary condition.
\\ \\
b) We see that combinations of $\cosh{n\pi y}$ and $\sinh{n\pi y}$ meet the requirements of $g_n(y,y')$.
We break the solution into the two regions $y<y'$ and $y>y'$. 
We must then solve for the coefficients using the boundary conditions, the requirement that $G(\bv{x},\bvp{x})$ is symmetric, and the requirement that $g(y=y')=-4\pi\delta(y-y')$.\\
We first write out the linear combinations split into the two regions:
\begin{align}
 g_n(y,y') = \begin{cases}
              A_n\sinh{(n\pi y')}+B_n\cosh{(n\pi y')},&y'<y\\
              C_n\sinh{(n\pi y')}+D_n\cosh{(n\pi y')},&y'>y\\
             \end{cases}
\end{align}
The boundary condition at $y'=0$ applies to the first case. 
It can be satisifed by taking all $B_n=0$.
The boundary condition at $y'=1$ applies to the second case, and results in the condition:
\begin{align}
 C_n\sinh{(n\pi (1))}+D_n\cosh{(n\pi (1))} &= 0\\
 \frac{-D_n}{C_n} &= \tanh{(n\pi)}\\
 D_n &= -\sinh{(n\pi)}\\
 C_n &= \cosh{(n\pi)}
\end{align}
Using the trig identity $\sinh{(a-b)}=\sinh{a}\cosh{b}-\cosh{a}\sinh{b}$ we can now write the full linear combination as:
\begin{align}
 g_n(y,y') = \begin{cases}
              A_n\sinh{(n\pi y')},&y'<y\\
              A'_n\sinh{(n\pi y' - n\pi)},&y'>y
             \end{cases}
\end{align}
We have inserted the coefficient $A'_n$ to recognize that our solution in the second region came from a ratio of coefficients, so it is undetermined up to some multiplier.\\
We now choose $A_n,A'_n$ to satisfy the requirement that $g_n(y,y')$ be symmetric. 
we note that the coefficients $A_n$ will have a functional form that satifies the symmetry, but also a common arbitrary constant which we call $F_n$.
\begin{align}
 g_n(y,y') = \begin{cases}
              F_n\lrb{\sinh{(n\pi y')}\times\sinh{(n\pi y - n\pi)}},&y'<y\\
              F_n\lrb{\sinh{(n\pi y' - n\pi)}\times\sinh{(n\pi y)}},&y'>y
             \end{cases}
\end{align}
To determine the $F_n$ normalization constants we examine the function at $y=y'$.
The second derivative must be a delta function at $y=y'$, so the first derivative must equal 1.
We therefore have the following equation for the $F_n$:
\begin{align}
 \Delta \frac{\partial g_n(y,y')}{\partial y}|_{y=y'} &= 1\\
 \frac{\partial g_n(y,y')}{\partial y}|_{y=y'} &= \begin{cases}
						  F_nn\pi\lrb{\sinh{(n\pi y-n\pi)}\times\cosh{(n\pi y)}},&y'<y\\
						  F_nn\pi\lrb{\sinh{(n\pi y)}\times\cosh{(n\pi y-n\pi)}},&y'>y\\
						\end{cases}\\
 F_nn\pi\{\sinh{(n\pi y-n\pi)}\cosh{(n\pi y)}-&\sinh{(n\pi y)}\cosh{(n\pi y-n\pi)}\} = 1\\
 F_nn\pi(\sinh{\lrp{n\pi y-n\pi y +n\pi}} ) &= 1\\
 F_n &= \frac{1}{n\pi\sinh{(n\pi)}}
\end{align}
We have now determined all coefficients, and the complete expansion is:
\begin{align}
\begin{split}
 G(x,x',y,y') = \int_{n=1}^\infty &\frac{1}{n\pi\sinh{(n\pi)}}\sin{(n\pi x)}\sin{(n\pi x')}\\
				  &\times\sinh{(n\pi y_<)}\sinh{\lrp{n\pi (1-y_>)}}
 \end{split}
\end{align}
Where we have adopted the usual convention of $y_<,Y_>$ as the lesser of (greater than) of $y$ and $y'$.

\end{document}
