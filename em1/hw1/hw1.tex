\documentclass[a4paper,11pt]{article}
\usepackage[utf8]{inputenc}
\usepackage{amsmath}
\usepackage{amsfonts}
\usepackage{amssymb}
\usepackage{graphicx}
\usepackage{braket}

\numberwithin{equation}{section}
\renewcommand\thesubsection{\alph{subsection}}
\newcommand{\bvp}[1]{\mathbf{#1}'}
\newcommand{\bv}[1]{\mathbf{#1}}


%opening
\title{Electromagnetic Theory I HW1}
\author{Vince Baker}

\begin{document}

\maketitle

\section{Problem 1.1}
a) Consider an element of excess charge inside the conductor.
Enclose the excess charge in a Gaussian sphere of arbitrary radius such that the surface of the sphere remains inside the conductor. 
By Gauss' Law there will be an electric field normal to the surface of the sphere, porportional to the encolsed excess charge.
However, any electric field inside a conductor will cause the free charges to move. 
Therefore, in equilibrium, any excess charge will be distributed on the surface of the conductor.
\\ \\
b)
\\ \\
c) On the surface of a conductor at equilibrium there must be no tangential component to the electric field, since nay tangential component would cause the free charge carriers to move.
To calculate the normal electric field due to the surface charge distribution take a Gaussian pillbox around a differential area element $da$.
Let the depth of the pillbox become arbitrarily small, so that the tangential components of the surface integral will be 0.
The total charge enclosed in the pillbox is $\sigma \  da$.
Gauss's law is then:
\begin{align}
 \int_{surface}\bv{E} \cdot d\bv{a} &= \sigma da
\end{align}



\section{Problem 1.4}

\end{document}
