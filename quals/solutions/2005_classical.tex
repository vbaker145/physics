\documentclass[a4paper,11pt]{article}
\usepackage[utf8]{inputenc}
\usepackage{amsmath}
\usepackage{amsfonts}
\usepackage{amssymb}
\usepackage{graphicx}
\usepackage{braket}

\numberwithin{equation}{section}
\renewcommand\thesubsection{\alph{subsection}}
\newcommand{\bvp}[1]{\mathbf{#1}'}
\newcommand{\bv}[1]{\mathbf{#1}}


%opening
\title{Drexel Physics 2005 Classical Qual Solutions}
\author{2014 entering class}

\begin{document}

\maketitle

\section{Problem 1}
We take the top of the triangle as the potential energy reference. 
Our one coordinate x is the displacement of $m_1$ from the top.
The kinetic and potential energy is:
\begin{gather}
 U=-g\{m_1x\sin{\alpha _1}+m_2(\ell-x)\sin{\alpha _2} \}\\
 K = \frac{1}{2}(m_1+m_2)\dot{x}^2
\end{gather}
The Lagrangian is then:
\begin{gather}
 L=K-U=\frac{1}{2}(m_1+m_2)\dot{x}^2+g\{m_1x\sin{\alpha _1}+m_2(\ell-x)\sin{\alpha _2} \}
\end{gather}
Calculating the partial derivatives we need for Lagrange's equations:
\begin{gather}
 \frac{\partial L}{\partial x}=g(m_1\sin{\alpha_1}-m_2\sin{\alpha_2})\\
 \frac{\partial L}{\partial \dot{x}}=(m_1+m_2)\dot{x}\\
 \frac{d}{dt}\frac{\partial L}{\partial \dot{x}}=(m_1+m_2)\ddot{x}
\end{gather}
Therefore, Lagrange's equation of motion for the system is:
\begin{gather}
 \frac{\partial L}{\partial x}-\frac{d}{dt}\frac{\partial L}{\partial \dot{x}}=0\\
 g(m_1\sin{\alpha_1}-m_2\sin{\alpha_2})-(m_1+m_2)\ddot{x}=0\\
 \ddot{x}-g(\frac{m_1}{m_1+m_2}\sin{\alpha_1}-\frac{m_2}{m_1+m_2}\sin{\alpha_2})=0
\end{gather}
\\
\section{Problem 2}
The astronaut's initial kinetic energy, $\frac{1}{2}mv^2$, is equal to his potential energy at the height of the jump, $m g_e h$.
So his velocity is $v=\sqrt{2g_e h}=2*\sqrt{\frac{GM_e}{r_e^2}}$.
The orbital velocity ($\sqrt{\frac{GM_d}{r_d}}$) and escape velocity ($\sqrt{\frac{2GM_d}{r_d}}$) for Deimos are porportional to the Earth's 
by $\frac{1}{r_e}$, since the mass is porportional to the radius cubed. 
Therefore the astronaut can reach both orbital and escape velocity from Deimos.
\\
\section{Problem 3}
The answer is (a) by Newton's third law. 
The force floating the magnet has an opposite force of equal magnitude pushing down on the cup.
\\
\section{Problem 4}
The fields must be perpendicular to each other and to the direction of propagation.
\\
\section{Problem 5}
a) Yes, the surface of a conductor is always an equipotential if it's at equilibrium.\\
b) The electric field lines that pass near the sphere curve into the surface. They are normal to the sphere at the point where they touch.\\
c): The field will cause negative charge to pile up at the +z pole, while positive charge piles up at the -z pole (taking the field as directed along z).
The charge distribution will be axially symmetric. You can find sketches online, and Griffiths uses this configuration to illustrate dipole moment.\\
d) Inside a conducting sphere there is no electric field.
\\
\section{Problem 6}
I would place them in a T configuration (the tip of one bar against the middle of the other).
The magnetized one will have the tip ``stick'' to the unmagnetized one.
\\
\section{Problem 7}
a) The wavelength is $\frac{c}{f}=0.125$ meters, or about 5 inches. 
There will be two full cycles in the standing wave.\\
b) Each water molecule will see an oscillating electric field which will set up oscillations in the polar water molecule. 
The zeros of the wave are the parts of the microwave oven that don't cook well, which is why your Hot Pocket$^{tm}$ can
burn your mouth while still having chunks of ice in some parts.\\
c) 
\\
\section{Problem B1}
For charge distribution $\rho=kr$ using Gauss' law:
\begin{gather}
 \int E \cdot da = \frac{q_{enc}}{\epsilon_0}\\
 4\pi r^2 E= \frac{1}{\epsilon_0}\int_V (kr) r^2 \sin{\theta} dr d\theta d\phi\\
 r^2 E = \frac{k}{4 \epsilon_0}r^4\\
 E = \frac{k}{4 \epsilon_0}r^2 \ \text{(inside sphere)}\\
 E = \frac{k}{4 \epsilon_0}\frac{R^4}{r^2} \ \text{(outside sphere)}
\end{gather}
\\
We calculate the energy of the configuration:
\begin{gather}
 W = \frac{\epsilon_0}{2}\int_V E^2 dV\\
 W = \frac{\epsilon_0}{2}\frac{k^2}{16 \epsilon_0^2}4\pi\{\int_0^Rr^6 dr+\int_R^\infty \frac{R^8}{r^2} dr\} \\
 W = \frac{\pi k^2}{8 \epsilon_0}\{\frac{R^7}{7}+R^7 \}\\
 W = \frac{\pi k^2}{7 \epsilon_0}R^7
\end{gather}
\\
The energy will approach 0 as $R \rightarrow 0$. 
The infinite-energy-at-0-radius applies to a sphere of CONSTANT total charge, in this case the charge is porportional to the radius.



\end{document}
