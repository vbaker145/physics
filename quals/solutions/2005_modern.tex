\documentclass[a4paper,11pt]{article}
\usepackage[utf8]{inputenc}
\usepackage{amsmath}
\usepackage{amsfonts}
\usepackage{amssymb}
\usepackage{graphicx}
\usepackage{braket}

\numberwithin{equation}{section}
\renewcommand\thesubsection{\alph{subsection}}
\newcommand{\bvp}[1]{\mathbf{#1}'}
\newcommand{\bv}[1]{\mathbf{#1}}


%opening
\title{Drexel Physics 2005 Modern Qual Solutions}
\author{2014 entering class}

\begin{document}

\maketitle

\section{Problem 1}
\section{Problem 3}
Electron tunneling in transistors.
\section{Problem A1}
We see that $S_x=\frac{S_+ + S_-}{2}$.
We find the eigenvalues in the usual manner:
\begin{gather}
 S_x = \frac{\hbar}{2}
 \begin{bmatrix}
  0 & \sqrt{2} & 0 \\
  \sqrt{2} & 0 & \sqrt{2} \\
  0 & \sqrt{2} & 0
 \end{bmatrix} \\
 \begin{vmatrix}
  -\lambda & \sqrt{2}/2 \hbar & 0 \\
  \sqrt{2}/2 \hbar & -\lambda & \sqrt{2}/2 \hbar \\
  0 & \sqrt{2}/2 \hbar & -\lambda
 \end{vmatrix} = 0\\
 -\lambda(\lambda^2-\hbar^2/2)-\hbar\sqrt{2}/2(-\lambda\hbar\sqrt{2}/2)=0\\
 -\lambda^3+\lambda \hbar^2=0\\
 \lambda = \{-\hbar, \hbar \}
\end{gather}
We now solve $S_x \bv{e} = \lambda \bv{e}$ for the eigenvectors.\\
\begin{gather}
 e_1=e_3\\
 e_2=\frac{1}{\lambda}\sqrt{2} \hbar e_1\\
 \lambda=-\hbar:\ \bv{e_{-\hbar}}=\{1,-\sqrt{2},1 \}\frac{1}{2}\\
 \lambda=\hbar:\ \bv{e_{\hbar}}=\{1,\sqrt{2},1 \}\frac{1}{2}
\end{gather}
We now  calculate the probability of measuring $S_x=\hbar$.
(The question says $S+x=1$, but 1 is not an eigenvalue of $S_x$.
\begin{gather}
 P(S_x=\hbar) = \braket{e_\hbar|u}^2= \left(\frac{1}{2\sqrt{2c}}(1*1+4*\sqrt{2}-3*1 )\right)^2\\
 P(S_x=\hbar) = \frac{1}{2c}(9-4\sqrt{2})
\end{gather}
After the measurement the system has a definite value of $S_x=\hbar$ and is in state $e_\hbar$.
The eigenvector corresponding to $S_z=\hbar$ is $(1,0,0)$, so we calculate the probability the same way.
\begin{gather}
 P(S_z=\hbar) = \braket{\bv{e_1}|\bv{e_\hbar}}^2= \left( 1*\frac{1}{2}+0+0 \right)^2\\
 P(S_z=\hbar) = \frac{1}{4}
\end{gather}
\\
\section{Problem A2}
a) We open up the brackets and find $\Psi(x,0) = 2x^3-10x^2_12x$.
\begin{gather}
 \int_0^3(C\Psi(x,0))^2 dx = 1\\
 \frac{1}{C^2}=\left(\int_0^32x^3-10x^2+12x dx \right)^2\\
 \frac{1}{C^2}=(-\frac{1}{2})^2=\frac{1}{4}\\
 C=2\\
 \Psi(x,0) = 2(2x^3-10x^2+12x)
\end{gather}
b) The wavefunction has a single 0 in the range at $x=2$, so it most closely resembles the standard wavefunction $\sin{\frac{2\pi}{3}x}$.
c) The wavefunctions can be derived from the Schrodinger equation:
\begin{gather}
 -\frac{\hbar^2}{2m}\frac{d^2 \psi}{dx^2}=e\psi\\
 \frac{d^2 \psi}{dx^2}=-\frac{2mE}{\hbar^2}\psi
\end{gather}
With $k \equiv \frac{\sqrt{2mE}}{\hbar}$, the solutions are of the form $A\cos{kx}+B\sin{kx}$.
The inifinte square well requires $\psi(0)=0$, so $A=0$.
This square well also requires $\psi(3)=0$, so k takes on discrete values determined by:
\begin{gather}
 \sin{\frac{n\pi x}{3}}=0\\
 k = \frac{n \pi}{3},\ n=1,2,3...\\
 \frac{\sqrt{2mE}}{\hbar}=\frac{n \pi}{3}\\
 E=\frac{n^2\pi^2 \hbar^2}{18m},\ n=1,2,3...
\end{gather}
So we can estimate the expectation value of the energy for $n=2$ as $\frac{2\pi^2\hbar^2}{9m}$.
\\
\section{Problem B1}
a) \\
i) Each distinguishable particle may be in one of 4 states, so there are 16 total states. 
Half the states have energy 0, half have energy $\epsilon$, so the partition function is:
\begin{gather}
 Z=\sum_1^8e^{-\beta\epsilon}+\sum_1^81\\
 Z=8(1+e^{-\beta\epsilon})
\end{gather}
ii) For Bosons there are only 8 possible states as the particles are indistinguishable. Once again, half the states have energy $\epsilon$.
\begin{gather}
 Z=\sum_1^4e^{-\beta\epsilon}+\sum_1^41\\
 Z=4(1+e^{-\beta\epsilon})
\end{gather}
iii) For Fermions the Pauli exclusion principle will eliminate the states with both particles having the same spin.
There are now 4 states, with half having energy $\epsilon$.
\begin{gather}
 Z=\sum_1^2e^{-\beta\epsilon}+\sum_1^21\\
 Z=2(1+e^{-\beta\epsilon})
\end{gather}
\\
b) As $T \rightarrow \infty$, $\beta=\frac{1}{kT}\rightarrow 0$ and $e^{-\beta\epsilon}\rightarrow 1$.
We note that the three partition functions can all be written as $Z=\frac{N}{2}(1+e^{-\beta \epsilon})$.
For each ensemble the probability of finding the system in a state with energy 0 is:
\begin{gather}
 f_0=\frac{e^{0}}{\frac{N}{2}\times 2}=\frac{1}{N}=f_\epsilon
\end{gather}
So half of the canonical systems for all types of particles will have energy 0 as $T\rightarrow \infty$.
\\
c) Writing the partition functions as $Z=\frac{N}{2}(1+e^{-\beta \epsilon})$ we find:
\begin{gather}
 \braket{E}=\sum E_i P_i=\frac{\sum E_i e^{-\beta E_i}}{Z}\\
 \braket{E}=\frac{\sum_1^{N/2}\epsilon e^{-\beta\epsilon}}{\frac{N}{2}(1+e^{-\beta \epsilon}})\\
 \braket{E}=\epsilon\frac{e^{-\beta\epsilon}}{1+e^{-\beta\epsilon}}\\
 \braket{E}=\epsilon\frac{1}{1+e^{\beta \epsilon}}
\end{gather}
\\
d) In the high temperature limit $\braket{E}=\frac{\epsilon}{2}$.

\section{Problem B2}
a) The heat capacity $C_v=(\frac{\partial U}{\partial T})_V$.
If the heat capacity is constant then $U\sim T$.
Then $A=U-TS=CT-TS=T(C-S)$ with C some constant.
b) For a monatomic ideal gas the Hamiltonian is $H=\frac{p^2}{2m}$.
The partition function of one atom is:
\begin{gather}
 Z_1=\frac{1}{h^3}\int d^3x\ d^3p\ e^{-\beta \frac{p^2}{2m}}\\
 Z_1 = \frac{V}{h^3}\int d^3p\ e^{-\beta \frac{p^2}{2m}}
\end{gather}
Each momentum component integral $\int_0^\infty e^{-\beta \frac{p^2}{2m}} dp=\sqrt{2m\pi kT}$.
With three independent components, the total partition function of one atom is:
\begin{gather}
 Z_1=\frac{V}{h^3}(2m\pi kT )^{3/2}\\
 Z_N=\frac{1}{N!}(\frac{V}{h^3})^N(2m\pi kT )^{3N/2}
\end{gather}
The Helmholtz free energy A is $A=-kT\ln{Z_N}$.
\begin{gather}
 A = -NkT(\ln{\frac{V}{N!}}+\frac{3}{2}\ln{\frac{2m\pi kT}{h^2}} )
\end{gather}
c) The energy levels of the harmonic oscillator are $(n+\frac{1}{2})\hbar \omega$. 
The partition function for one atom is:
\begin{gather}
 \sum_{s=0}^\infty e^{-\beta(s+\frac{1}{2})\hbar \omega}=e^{-\beta\frac{1}{2}\hbar\omega}(1+e^{-\beta\hbar\omega}+e^{-2\beta\hbar\omega}+...)\\
 =e^{-\beta\frac{1}{2}\hbar\omega}(\frac{1}{1-e^{\beta\hbar\omega}})
\end{gather}
The partition function for N atoms is $Z_1^N=\left(e^{-\beta\frac{1}{2}\hbar\omega}(\frac{1}{1-e^{\beta\hbar\omega}})\right)^N$.
The Helmholtz free energy is:
\begin{gather}
 A=-kT\ln{Z_N}\\
 A=-NkT(-\beta\frac{1}{2}\hbar\omega+\beta\hbar\omega)\\
 A=-\frac{N}{2}\hbar\omega
\end{gather}

\end{document}
