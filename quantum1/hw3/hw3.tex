\documentclass[a4paper,10pt]{article}
\usepackage[utf8]{inputenc}
\usepackage{amsmath}
\usepackage{graphicx}
\numberwithin{equation}{section}

%opening
\title{}
\author{Vince Baker}

\begin{document}

\maketitle

\begin{abstract}
Quantum 1 HW 3
\end{abstract}

\section{Problem 1}
To calculate the Compton effect we look at relativistic scattering of a photon from an electron following notes from Dr. McCray's Classical Dynamics course. 
We use both conservation of energy and conservation of momentum.
We define the momentum of the incident photon as $p_i$ and the momentum of the scattered photon as $p_s$.
The electron is initially at rest, and after scattering has momentum $p_e$.
Prior to scattering, the photon has energy $p_ic$ and the electron has energy $m_ec^2$. 
After scattering the scattered photon and electron obey conservation of energy.
\begin{gather}
 p_ic+m_ec^2=p_sc+\sqrt{(p_e^2c^2+m_e^2c^4)}\\
 p_e^2c^2=(p_i-p_s)^2c^2+2c(p_i-p_e)m_ec^2\\
 p_e^2=(p_i-p_s)^2+2c(p_i-p_s)m_e
\end{gather}
We also use conservation of momentum. The electron initially has no momentum.
The vector sum of the scattered photon momentum and the electron momentum must equal the original photon momentum.
Using the law of cosines with $\theta$ the scattering angle of the photon:
\begin{gather}
 p_e^2=p_i^2+p_s^2-2p_ip_scos(\theta)
\end{gather}
We can now use 1.3 and 1.4 together:
\begin{gather}
 p_i^2+p_s^2-2p_ip_scos(\theta)=(p_i-p_s)^2+2c(p_i-p_s)m_e=p_i^2-2p_ip_s+p_s^2+2cm_e(p_i-p_s)\\
 -2p_ip_scos(\theta)=-2p_ip_s+2cm_e(p_i-p_s)\\
 p_ip_s(1-cos(\theta))=cm_e(p_i-p_s)
\end{gather}
The photon momentum is equal to $\frac{\hbar}{\lambda}$, so we can write:
\begin{gather}
 \frac{\hbar ^2}{\lambda_i \lambda_s}(1-cos(\theta))=m_ec\hbar (\frac{1}{\lambda_i}-\frac{1}{\lambda_s}\\
 \lambda_s - \lambda_i = \frac{\hbar}{m_ec}(1-cos(\theta))
\end{gather}
A photon that scatters from an electron will see an average change in wavelength of $\frac{\hbar}{m_ec}$, the Compton wavelength of the electron.

\section{Problem 2}
The energy to assemble a charge distribution is:
\begin{equation}
 W=\frac{1}{2} \int \sigma V da
\end{equation}
The potential at the surface of the uniformly charged sphere is $\frac{1}{4\pi \epsilon_0}\frac{-e}{R}$.
\begin{gather}
 W=\frac{1}{8\pi \epsilon_0}\frac{-e}{R}\int \sigma da= \frac{1}{8\pi \epsilon_0}\frac{e^2}{R}
\end{gather}
We now equate this energy to the rest mass of the electron and caculate the classical electron radius.
\begin{gather}
  \frac{1}{8\pi \epsilon_0}\frac{e^2}{R}=m_ec^2\\
  R_e=\frac{1}{8\pi \epsilon_0}\frac{e^2}{mc^2}
\end{gather}

\section{Problem 3}
Trying simple combinations of two of the three physical constants, we have:
\begin{gather}
 QM \bigcup EM: \frac{e^2}{\hbar}=\frac{L}{T}\\
 QM \bigcup SR: \frac{\hbar}{c}=ML\\
 EM \bigcup SR: \frac{e^2}{c^2}=ML
\end{gather}
To find a unit of length, we really require all three physical constants, and we get the fine structure constant.
\begin{gather}
 \alpha = \frac{e^2}{\hbar c} \ (dimensionless)
\end{gather}

\section{Problem 4}
The Schrodinger equation of a fixed rotor with angular momentum $I_0$ is:
\begin{gather}
 \frac{-\hbar ^2}{2I_0}\frac{d^2 \Psi}{d\phi ^2}=E\Psi\\
 \frac{d^2 \Psi}{d\phi ^2}+\frac{2I_0E}{\hbar ^2}\Psi=0
\end{gather}
The solutions are $\Psi_m=e^{i m \phi},\ m=0,\pm 1, \pm2...$, where the restriction to integer values of m comes from the requirement that $\Psi(\phi)$ be single-valued.
The separation constant $m^2=\frac{2I_0E}{\hbar ^2}$ allows us to calculate the energy levels:
\begin{equation}
 E_m=\frac{m^2 \hbar^2}{2I_0},\ m=0,\pm 1, \pm 2...
\end{equation}
$E_m \leq 100\frac{\hbar ^2}{2I_0}$ when $|m|\leq10$. There are therefore 21 states with $E_m \leq 100\frac{\hbar ^2}{2I_0}$.
\\ \\
With a variable axis we are now working with a full spherical coordinate system. The Schrodinger equation is:
\begin{gather}
 \frac{-\hbar^2}{2I_0}\nabla^2\Psi=E\Psi\\
 \nabla^2\Psi+\frac{}{}\Psi=0
\end{gather}
With a fixed rotor length we can ignore the radial part of this solution. 
The angular part of the solution is a spherical harmonic $Y_\ell^m(\theta, \phi)$.
This will consist of an exponential part in $\phi$ and an associated Legendre function in $\theta$.
The exponential part gives rise to the constraint that m is an integer.\\ \\
The angular part will be equal to a separation constant, which we set to $-\ell(\ell+1)$ by convention. 
We can then relate this to the expression for the fixed-axis rotor energies.
\begin{gather}
 \nabla^2Y_\ell^m(\theta, \phi)=-\ell(\ell+1)\\
 E=\frac{\ell(\ell+1)\hbar^2}{2I_0}
\end{gather}
We apparently recognize that this is $(2\ell+1)$-fold degenerate. 
The maximum value of $\ell$ that satisfies $E_m \leq 100\frac{\hbar ^2}{2I_0}$ is 9.
With 9 energy levels and $(2\ell+1)$-fold degeneracy there are 99 states with energy $E_m \leq 100\frac{\hbar ^2}{2I_0}$.

\end{document}
