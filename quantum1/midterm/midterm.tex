\documentclass[a4paper,10pt]{article}
\usepackage[utf8]{inputenc}
\usepackage{amsmath}
\usepackage{graphicx}
\numberwithin{equation}{section}

%opening
\title{Quantum 1 Midterm, Klein-Gordon equation}
\author{Vince Baker}

\begin{document}

\maketitle

\section{Klein-Gordon equation}
We wish to determine the relativistic energy levels of hydrogen-like atoms.
We consider the relativistic energy of a free particle:
\begin{gather}
 E^2=(mc^2)^2+(pc)^2\\
 (pc)^2+(mc^2)^2-E^2=0
\end{gather}
We can convert this to a second-order PDE by using Schrodinger's substitution $p\rightarrow\frac{\hbar}{i}\nabla$.
We now have a second-order operator acting on a wavefunction $\psi$:
\begin{gather}
\{(\frac{\hbar c}{i}\nabla)^2+(mc^2)^2-E^2\}\psi = 0
\end{gather}
We now put our free particle in a scalar potential V. 
This will be the electrostatic potential from the nuclear charge.
Going back to 1.1, we substitute $(E-V)$ for $E$.
The negative sign retains conservation of potential/kinetic energy (as potential energy goes up, the kinetic energy goes down).
\begin{gather}
 \{(\frac{\hbar c}{i}\nabla)^2+(mc^2)^2-(E-V)^2)\}\psi = 0\\
 -(\hbar c)^2\nabla ^2\psi+\{(mc^2)^2-(E-V)^2\}\psi=0
\end{gather}
The Laplacian in three dimensions is:
\begin{gather}
 \nabla^2\chi(r,\phi,\theta)=\frac{1}{r^2}\frac{\partial}{\partial r}(r^2\frac{\partial \chi}{\partial r})+\frac{1}{r^2\sin \theta}\frac{\partial}{\partial \theta}(\sin \theta \frac{\partial \chi}{\partial \theta} )
 +\frac{1}{r^2 \sin ^2 \theta}\frac{\partial ^2 \chi}{\partial \phi^2}
\end{gather}
We assume we can separate 1.5 into $\psi(r,\theta, \phi)=R(r)\Theta(\theta)\Phi(\phi)$. We can then use 1.6 to write 1.5 as:
\begin{multline}
  -(\hbar c)^2\{\frac{1}{r^2}\frac{\partial}{\partial r}(r^2\frac{dR}{dr})\Phi\Theta+\frac{1}{r^2\sin \theta}\frac{\partial}{\partial \theta}(\sin \theta \frac{d\Theta}{d \theta})R\Phi\\
 +\frac{1}{r^2 \sin ^2 \theta}\frac{d^2\Phi}{d \phi^2}R\Theta\}
 +\{(mc^2)^2-(E-V(r))^2\}R\Phi\Theta=0
\end{multline}
Dividing through by $\frac{-(\hbar c)^2R\Theta\Phi}{r^2}$ we have:
\begin{gather}
 \frac{1}{R}\frac{d}{dr}(r^2\frac{dR}{dr})+\frac{1}{\Theta\sin \theta}\frac{d}{d \theta}(\sin \theta \frac{d\Theta}{d \theta})
 +\frac{1}{\Phi \sin ^2 \theta}\frac{d^2\Phi}{d \phi^2}
 -r^2\frac{(mc^2)^2-(E-V(r))^2}{(\hbar c)^2}=0
\end{gather}
The first and fourth terms are expressions in r only, the second and third terms are expressions in $\theta$ and $\phi$.
We can therefore separate the angular and radial terms.
The solutions to the angular terms are the spherical harmonics $P_{\ell}^{m}(\cos \theta)e^{im\phi}$.
We examine the radial solution by setting the separation constant to $\ell(\ell+1)$, and we have:
\begin{gather}
  \frac{1}{R}\frac{d}{dr}(r^2\frac{dR}{dr})-r^2\frac{(mc^2)^2-(E-V(r))^2}{(\hbar c)^2}=\ell(\ell+1)\\
  \frac{d}{dr}(r^2\frac{dR}{dr})+\frac{Rr^2}{(\hbar c)^2}\{(E-V(r))^2-(mc^2)^2)\}-R(\ell(\ell+1))=0
\end{gather}
We make the substitution $R=\frac{u}{r}$. We then find:
\begin{gather}
R = \frac{u}{r}\\
\frac{dR}{dr}=\frac{u^{'}}{r}-\frac{u}{r^2}= \frac{ru^{'}-u}{r^2}\\
\frac{d}{dr}(r^2\frac{dR}{dr})=\frac{d}{dr}(ru^{'}-u)=ru^{''}
\end{gather}
We can now substitute into 1.10:
\begin{gather}
 r\frac{d^2u}{dr^2}+\frac{ur}{(\hbar c)^2}((E-V(r))^2-(mc^2)^2))-\frac{u}{r}(\ell(\ell+1))=0\\
 \frac{d^2u}{dr^2}+\frac{u}{(\hbar c)^2}((E-V(r))^2-(mc^2)^2))-\frac{u}{r^2}(\ell(\ell+1))=0
\end{gather}
We now introduce a dimensionless constant $r=\gamma z$ so that we can look up the solution for this PDE.
This is also a good time to introduce our potential function, $V(r) = \frac{-e^2}{r}$.
\begin{gather}
 \frac{1}{\gamma ^2}\frac{d^2u}{dz^2}+\frac{u}{(\hbar c)^2}((E-\frac{-e^2}{\gamma z})^2-(mc^2)^2))-\frac{u}{(\gamma z)^2}\ell(\ell+1)=0\\
 \frac{1}{\gamma ^2}\frac{d^2u}{dz^2}+\frac{u}{(\hbar c)^2}((E^2+2E\frac{e^2}{\gamma z}+(\frac{-e^2}{\gamma z})^2   )-(mc^2)^2))-\frac{u}{(\gamma z)^2}\ell(\ell+1)=0\\
 \frac{d^2u}{dz^2}+\{\frac{1}{z}\frac{2Ee^2\gamma}{(\hbar c)^2}+\frac{-\ell(\ell+1)+\frac{e^4}{(\hbar c)^2})}{z^2}+\frac{\gamma^2}{(\hbar c)^2}(E^2-(mc^2)^2) \}u=0
\end{gather}
We find a solution of the right form in Abramowitz and Stegun:
\begin{gather}
 \frac{d^2y}{dx^2}+\{\frac{2n+\beta +1}{2x}+\frac{1-\beta ^2}{4x^2}-\frac{1}{4} \}y=0
\end{gather}
Equating terms and making the substitution $\alpha=\frac{e^2}{(\hbar c)^2}$, we find:
\begin{gather}
  -\ell(\ell+1)+\alpha^2=\frac{1-\beta ^2}{4}\\
  2E\alpha \frac{\gamma}{\hbar c} =\frac{2n+\beta +1}{2}\\
 \frac{\gamma^2}{(\hbar c)^2}(E^2-(mc^2)^2)=-\frac{1}{4}
\end{gather}
From the first equation we see that $4[(\ell+\frac{1}{2})^2-\alpha ^2]=\beta^2$, so $\beta=2\sqrt{(\ell+\frac{1}{2})^2-\alpha ^2}$.
We use the second equation to solve for $\frac{\gamma}{\hbar c}$, then plug into the third equation to find an expression for E.
\begin{gather}
 2E\alpha \frac{\gamma}{\hbar c}=\frac{2n+\beta+1}{2}\\
 2E\alpha \frac{\gamma}{\hbar c}=n+\sqrt{(\ell+\frac{1}{2})^2-\alpha^2}+\frac{1}{2},\ 
 N(\alpha) \equiv n+\sqrt{(\ell+\frac{1}{2})^2-\alpha^2}+\frac{1}{2}\\
 \frac{\gamma}{\hbar c}=\frac{N(\alpha)}{2E\alpha}\\
 (\frac{N(\alpha)}{2E\alpha})^2(E^2-(mc^2)^2)=-\frac{1}{4}\\
 -1+\frac{(mc^2)^2}{E^2}=(\frac{\alpha}{N(\alpha)})^2\\
 E^2=\frac{(mc^2)^2}{1-(\frac{\alpha}{N(\alpha)})^2}\\
 E=\frac{mc^2}{\sqrt{1-(\frac{\alpha}{N(\alpha)})^2}}
\end{gather}
We can now substitute E into equation 1.21 to find $\gamma$.
\begin{gather}
 2E\alpha \frac{\gamma}{\hbar c}=N(\alpha)\\
 \gamma = \frac{\hbar c}{2mc^2}\sqrt{1-(\frac{\alpha}{N(\alpha)})^2}\frac{N(\alpha)}{\alpha}\\
 \gamma = \frac{\hbar c}{2mc^2}\sqrt{(\frac{(N(\alpha)}{\alpha})^2-1}
\end{gather}
The presence of minus signs under square roots raises a question of imaginary values for the energy spectra. 
We examine the ground state, $n=\ell=0$, so that $N(\alpha)=\frac{1}{2}+\sqrt{\frac{1}{4}-\alpha^2}$.
Since the fine structure constant $\alpha\approx\frac{1}{137}$, this works fine for hydrogen.
However it may fail for other atoms with a higher nuclear charge $Ze$ (Z an integer).
A nuclear charge of $Ze$ will modify the expression for $\alpha$ to $\frac{(Ze)-e}{\hbar c}=Z\alpha$. 
We then find the nuclear charge at which $N(Z\alpha)$ becomes complex.
\begin{gather}
 \frac{1}{4}-Z^2\alpha ^2=0\\
 Z^2=\frac{1}{4\alpha^2}\approx\frac{137^2}{2^2}\\
 Z\approx \frac{137}{2}=68.5
\end{gather}
So for nucleii with atomic number 69 (thulium) or greater the Klein-Gordon equation will produce complex numbers for the energy spectrum.
\\
The rest energy $mc^2$ will typically dominate the total energy.
Returning to equation 1.2, we expand $E=(mc^2+W)$ as the rest energy plus a nonrelativistic energy W. 
We then expand $(E-V)^2$:
\begin{gather}
 (mc^2+W-V)^2=(mc^2+(W-V))^2=(mc^2)^2+2mc^2(W-V)+(W-V)^2
\end{gather}
We then plug this into 1.2:
\begin{gather}
 (pc)^2+(mc^2)^2-E^2=0\\
 (pc)^2+(mc^2)^2-\{(mc^2)^2+2mc^2(W-V)+(W-V)^2\}=0\\
 (pc)^2-2mc^2(W-V)-(W-V)^2=0\\
 \frac{p^2}{2m}-(W-V)-\frac{(W-V)^2}{2mc^2}=0
\end{gather}
In the nonrelativistic limit we can discard the third term and we recover the classical result (replacing W by E).
\begin{gather}
 \frac{p^2}{2m}+V=E
\end{gather}


\end{document}
